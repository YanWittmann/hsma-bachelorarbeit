\chapter{Evaluation}\label{ch:evaluation}


\section{Evaluationsmethodik}


\section{Ergebnisse}


\section{Diskussion}


% \subsection{Zusammenfassung und Ausblick}\label{subsec:impl-summary}

% Die Implementierung des neuen Korrelationssystems als Graphdatenbank mit spezialisierten Knotentypen und Beziehungen ermöglicht eine flexible und erweiterbare Lösung für die Korrelation von Softwareartefakten mit verschiedenen Repräsentationen.
% Durch die Verwendung von SQLite als Datenbanktechnologie wird eine einfache Verteilung und Verwaltung der Daten ermöglicht, während die Implementierung in Java eine nahtlose Integration in das bestehende Schwachstellenmanagement gewährleistet.

% Das System bietet mehrere Vorteile gegenüber dem bisherigen Ansatz:
% \begin{itemize}
%     \itemsep0em
%     \item Eine klare Trennung zwischen verschiedenen Repräsentationstypen durch spezialisierte Knotenklassen
%     \item Flexible Erweiterbarkeit durch das Hinzufügen neuer Knotentypen und Beziehungen
%     \item Verbesserte Nachvollziehbarkeit durch explizite Begründungen für Korrelationsentscheidungen
%     \item Effiziente Abfragen durch die Verwendung einer Graphstruktur und Caching-Mechanismen
%     \item Einfache Wartung und Aktualisierung durch das YAML-basierte Modifikationsformat
% \end{itemize}

% Für zukünftige Erweiterungen des Systems sind mehrere Möglichkeiten denkbar:
% \begin{itemize}
%     \itemsep0em
%     \item Integration weiterer Repräsentationstypen wie SWID-Tags oder andere Identifikationsstandards
%     \item Entwicklung einer grafischen Benutzeroberfläche für die Visualisierung und Bearbeitung des Graphen
%     \item Automatisierte Validierung der Graphkonsistenz und Erkennung von Widersprüchen
%     \item Erweiterung der Abfragesprache für komplexere Traversierungen und Filterungen
% \end{itemize}

% Insgesamt bietet die Implementierung eine solide Grundlage für die Weiterentwicklung des Korrelationssystems und die Integration in das bestehende Schwachstellenmanagement.
