\section{Referenzfälle}\label{sec:reference-case-chapter}

Aus einer Auswertung aus dem bereits vorhandenen Korrelationsdatensatz wurden vier grundlegende Fälle identifiziert, die in ihrer Kombination den Großteil der Komplexität in den bisherigen Daten abdecken.
Sie wurden nicht aufgrund von ihren Inhalten oder betreffenden Produkten gewählt, sondern wegen ihrer Struktur und verwendeten impliziten oder expliziten Mechanismen.
Das neue Modell muss also mindestens diese Fälle abdecken können und daher werden sie als Referenzfälle verwendet.

Neben dem konzeptionellen Unterschied zum neuen Korrelationsformat der Modellierung als Graphen haben diese Beispiele im alten Format zudem die Eigenschaft, dass sie mit dem Ziel erstellt werden, ein Inventar-Artefakt zu modifizieren.
Es wird also immer ausschließlich von einem Artefakt ausgegangen, und dessen Ziel-Attribut-Werte definiert.
Im neuen Format wird dieser Gedanke in den Hintergrund treten, und sich implizit durch den Aufbau des Graphen ergeben.

Es werden zunächst immer die Einträge vorgestellt, dann die Herausforderungen und relevanten Anforderungen identifiziert.
Diese Referenzfälle werden in \autoref{sec:beispiele-fertige-implementierung} jeweils auf das neue Korrelationssystem übersetzt, um einen Vergleich der Formate zu ermöglichen.

% FIXME-KKL: Zusammenhalten: Ich glaube der Kommentar hier bezieht sich darauf, dass im PDF zwischen dem Titel und der Klammer ein Abstand ist. Den habe ich schon versucht wegzubekommen, aber LaTex will das unbedingt so aufteilen.
\paragraph{JavaScript-Paket mit multiplen Namensvarianten}\label{par:reference-case-walletconnect} (s. \autoref{lst:correlation-generated-walletconnect-example})

Bei dem Paket \texttt{@web3-react/walletconnect} handelt sich in der NPM-Registry zwar um ein eindeutiges Paket, allerdings kann sich der Bezeichner in den gescannten Software-Inventaren in unterschiedlichen Varianten ausprägen, auf die alle geprüft werden muss.
In diesem Fall handelt es sich um eine Positividentifikation auf einer \acrshort{cpe} \texttt{cpe:/a:uniswap:web3-react\_walletconnect}, die als \texttt{Additional CPE URIs} zum Artefakt hinzugefügt wird.
Um die drei alternative Namensformen in dem Selektor in der \texttt{affects}-Sektion erfassen zu können, müssen diese alle separat mit Wiederholung der restlichen Attribute in alternativen Selektoren geführt werden.
Über dem Eintrag ist ein \texttt{reason}-Kommentar, der eine Referenz auf das Paket auf NPM und auf die Homepage des Herstellers enthält.

Dieser Fall, wie auch die folgenden, demonstriert die Notwendigkeit, ökosystemspezifisch-normalisierte Bezeichner zu extrahieren (\hyperref[subsec:req-type-specific-matching]{A-ARTIFACT-TYPES-IDENTIFICATION}).
Die \enquote{-*}-Suffixe können zu einer Übergeneralisierung der Einträge führen, aber auch der häufig wiederholte reguläre Ausdruck \texttt{/(web|nodejs|npm)-module/} für die Erkennung des Artefakt-Typen kann zu Fehlern führen und stellt die üblichen Probleme von Redundanz dar.

\hyperref[subsec:req-multiple-attribute-values]{A-MULTIPLE-VALUES-PER-ATTRIBUTE} ist aus der Notwendigkeit entstanden, innerhalb eines Eintrages die Wiederholung nah identischer Selektoren zu vermeiden.
In diesem Fall war es nötig, für jede Namensform einen neuen Selektor zu erstellen, und alle weiteren Attribute zu wiederholen.

\paragraph{Java-Runtimes: Komplexe Versionstransformation}\label{par:reference-case-java-runtimes} (s. \autoref{lst:reference-case-java-runtimes})

Wie in \autoref{subsec:old-generated-correlation-data} bereits gezeigt, wird ein wesentlicher Teil des alten Korrelationsdatensatzes regelmäßig automatisch in einem dedizierten Prozess generiert.
Das Listing zeigt einen Auszug der generierten Einträge für die Java-Runtime \enquote{Amazon Corretto} und \enquote{Azul Zulu}.
Diese wurden einmal aus der Notwendigkeit eingeführt, da die \acrshortpl{cpe} der Java Runtimes einige der wenigen sind, bei denen häufig der Update-Part tatsächlich verwendet wird und andererseits die Versionen der \acrshortpl{cpe} selbst innerhalb eines Vendor-Product-Namensraums sehr uneinheitlich strukturiert sein können.
Einige häufige Beispiele für das Format der Version- und Update Parts sind: \texttt{21.0.6:*}, \texttt{23:*}, \texttt{*:update32}, \texttt{*:update\_32}, \texttt{1.6.0:update32\_b31}, \texttt{1.6.0:update32\_b32}.
Für jede dieser Versionen ist ein neuer Eintrag in den \acrshort{yaml}-Dateien nötig, da jeder Eintrag nur eine versionierte \acrshort{cpe} adressieren kann.

Die zugehörige Anforderung für die automatische Befüllung des Graphen ist \hyperref[subsec:req-generated-data]{A-GENERATION-FRAMEWORK}.
Die Anforderung \hyperref[subsec:req-type-specific-matching]{A-ARTIFACT-TYPES-IDENTIFICATION} trifft hier ebenfalls zu, denn die Erkennung des Artefakts als Java Runtime, des Anbieters und die Versionserkennung muss implizit an derselben Stelle in der \texttt{Id} passieren.
Die Erkennung von drei Metriken über ein einziges Artefakt-Attribut ist weder zuverlässig, noch ist es gut auf weitere Formate der \texttt{Id} erweiterbar.
In dem neuen Format sollte die Identifikation von diesen zusammengesetzen Attributen durch programmatische Unterstützung in der Selektorlogik auf einzelne Attribute aufgetrennt werden.
Mit der Änderung des Formats als Graphen (\hyperref[subsec:req-format-product-graph]{A-GRAPH-MODEL}), in dem jede Repräsentation versions-agnostisch nur einmal als Knotenpunkt ausgeprägt ist und mit der Anforderung, dass es ein ausgeprägtes Produkt-Konzept mit Produkt-Knotenpunkten gibt (\hyperref[subsec:req-product-concept]{A-PRODUCT-CONCEPT}), kann die Versionserkennung und Modifikation der Version in der \acrshort{cpe} in den einzelnen Knotenpunkten der Repräsentationen nicht mehr wie früher modelliert werden, sondern muss auf einen zentral die Repräsentationen verknüpenden Produkt-Knotenpunkt, der diese Mappings unter einem Namensraum an die Knotenpunkte zur Verfügung stellt.
Zur Erkennung der Version sollte mindestens noch immer das Wildcard-System zur Verfügung gestellt werden (\hyperref[subsec:req-regex-support]{A-REGULAR-EXPRESSIONS}).

% FIXME-KKL: /zulu.*-(?:jre|jdk)-headless-11\.0\.10.*/i ist nicht korrekt: was wäre denn dann richtig? so steht es in den korrelationsdaten drin.
\begin{lstlisting}[style=yaml,caption={Java-Runtime-Korrelation mit Versionstransformation},label={lst:reference-case-java-runtimes},basicstyle=\ttfamily\scriptsize]
- affects:
    - Id: amazon-corretto-*/i
    - Id: /java-(?:\d\.?)+-amazon-corretto-.*/i
  append:
    Additional CPE URIs: cpe:/a:amazon:corretto, cpe:/a:oracle:jdk, cpe:/a:oracle:jre
- affects:
    - Id: /amazon-corretto-1\.1\.6.*9.*/i
    - Id: /amazon-corretto-java-1\.1\.6.*9.*/i
    - Id: /java-(?:\d\.?)+-amazon-corretto(?:-jdk)?-1\.1\.6.*9.*/i
  append:
    CPE URIs: cpe:/a:oracle:jre:1.1.6_009, cpe:/a:amazon:corretto

- affects:
    - Id: /zulu.*-(?:jre|jdk)-headless-.*/i
    - Id: /zulu.*-(?:jre|jdk)-.*/i
  append:
    Additional CPE URIs: cpe:/a:azul:zulu
- affects:
    - Id: /zulu.*-(?:jre|jdk)-headless-11\.0\.10.*/i
    - Id: /zulu.*-(?:jre|jdk)-11\.0\.10.*/i
  append:
    CPE URIs: cpe:/a:azul:zulu:11.0.10
- affects:
    - Id: /zulu.*-(?:jre|jdk)-headless-1\.8\.0.*[^0-9]282.*/i
    - Id: /zulu.*-(?:jre|jdk)-1\.8\.0.*[^0-9]282.*/i
  append:
    CPE URIs: cpe:/a:azul:zulu:8:update282
\end{lstlisting}

\paragraph{Redis: Kontextabhängige Produktdifferenzierung}\label{par:reference-case-redis} (s. \autoref{lst:correlation-order-depdendency-example})

Der Redis-Fall zeigt die Herausforderung von ökosystemabhängigen überladenen Artefaktbezeichnern.
Der gleiche Basisname \texttt{redis-*} repräsentiert in den Referenzeinträgen aus den Software-Inventaren mindestens zwei unterschiedliche Entitäten: die Datenbankserver-Software und die Client-Bibliothek als Python-Modul.
Im alten Format erfordert dies entweder eine Wiederholung des Artefakt-Selektors in dem spezifischeren Eintrag als \texttt{ignores}, oder eine \texttt{remove}-Ausschlussregel mit der die generelle Modifikation wieder rückgängig gemacht wird.
Dies ist durch die bedingte Reihenfolge fehleranfällig und die Entscheidungslogik kann nicht zureichend dokumentiert werden.

Die klare Trennung zwischen Produkten und Repräsentationen (\hyperref[subsec:req-product-concept]{A-PRODUCT-CONCEPT}) im Graphenmodell soll eine bessere Ausprägung der Repräsentationen ermöglichen.
Durch die \enquote{is}/\enquote{is not}-Relationen soll der spezifischste Knotenpunkt explizit angeben können, wie mit einer referenzierten Repräsentation umgegangen werden soll (\hyperref[subsec:req-format-product-graph]{A-GRAPH-MODEL}), und strukturierte Metadaten (\hyperref[subsec:req-reason-format]{A-REASON-RATIONALE}) sollen die Dokumentation maschinenlesbar machen.

\paragraph{Windows 10: Betriebssystem mit mehreren Identifikatoren}\label{par:reference-case-windows} (s. \autoref{lst:reference-case-windows})

Die Modellierung von Windows 10 in den alten Korrelationsdaten stellt einige größere Herausforderungen dar.
Da Windows als Microsoft-Produkt sich nicht nur als \acrshort{cpe} repräsentiert, sondern auch durch Microsoft Produkt-Ids und zusätzlich noch EOL Ids an die Artefakte angehängt werden sollen, müssen für unterschiedliche Versionen des Betriebssystems auch immer unterschiedliche Repräsentationen verknüpft werden.
Dies ist problematisch, da die Basisinformationen auf allen Windows-Artefakten dieselbe ist, und diese dann für jede zusätzliche zu unterstützende Version in den Artefakt-Selektoren wiederholt werden muss.
Selbst innerhalb eines Eintrages müssen die Selektoren wiederholt werden, um die unterschiedlichen Ausprägungen der \texttt{Id} zu unterstützen.
Dies führt zu einer großen Menge an Redundanzen.
Zudem ist die Hierarchie der Einträge (Windows 10, 21H2, Architektur) nur impliziert modelliert und muss von einem Leser erst als solche erkannt werden.

Die relevanten Anforderungen sind die Vererbung von Artefakt-Selektoren (\hyperref[subsec:req-selektor-inheritance]{A-ARTIFACT-INHERITANCE}), die die Wiederholung über Einträge hinweg vermeidet, und die Auflistung mehrerer Werte pro Attribut (\hyperref[subsec:req-multiple-attribute-values]{A-MULTIPLE-VALUES-PER-ATTRIBUTE}).
Zudem ist hier besonders die Dokumentation der unterschiedlichen Knotenpunkte und Kanten relevant (\hyperref[subsec:req-reason-format]{A-REASON-RATIONALE}).

\begin{lstlisting}[style=yaml,caption={Windows-Korrelation mit mehreren Identifikatoren},label={lst:reference-case-windows},basicstyle=\ttfamily\scriptsize]
- affects:
    - Id: Windows 10*
      Type: operating system
    - Id: Microsoft Windows 10*
      Type: operating system
  remove:
    Additional CPE URIs: cpe:/o:microsoft:windows
  append:
    Inapplicable CPE URIs: cpe:/a:windows:media_player, cpe:/o:mircorsoft:windows, cpe:/o:microsoft:windows
    Additional CPE URIs: cpe:/o:microsoft:windows_10
    EOL Id: windows

# reason: https://learn.microsoft.com/de-de/windows/release-health/release-information
#         11931 --> Version 21H2 (OS build 19044) / Windows 10 Version 21H2 for x64-based Systems
- affects:
    - Id: Microsoft Windows 10*
      Version: 10.0.19044*
      Type: operating system
      Architecture: "*64*"
    - Id: Windows 10*
      Version: 10.0.19044*
      Type: operating system
      Architecture: "*64*"
  append:
    MS Product ID: "11931"

# reason: 11929 --> Version 21H2 (OS build 19044) / Windows 10 Version 21H2 for 32-bit Systems
- affects:
    - Id: Microsoft Windows 10*
      Version: 10.0.19044*
      Type: operating system
      Architecture: "*32*"
    - Id: Windows 10*
      Version: 10.0.19044*
      Type: operating system
      Architecture: "*32*"
  append:
    MS Product ID: "11929"

# reason: 11929 --> Version 21H2 (OS build 19044) / Windows 10 Version 21H2 for 32-bit Systems
- affects:
    - Id: Microsoft Windows 10*
      Version: 10.0.19044*
      Type: operating system
  append:
    Additional CPE URIs: cpe:/o:microsoft:windows_10_21h2, cpe:/o:microsoft:windows_10:21h2
\end{lstlisting}
