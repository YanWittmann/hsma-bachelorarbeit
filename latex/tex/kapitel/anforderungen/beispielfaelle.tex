\section{Referenzfälle}\label{sec:reference-case-chapter}

Aus einer Analyse des bestehenden Korrelationsdatensatzes wurden vier Referenzfälle ausgewählt, welche die identifizierten Herausforderungen der bisherigen Modellierung abdecken.
Deren Abbildung auf das neue System mit einer Verbesserung der Modellierung wird als Mindestanspruch vorausgesetzt.
Die Auswahl erfolgte auf Basis der strukturellen Merkmale und einer Abdeckung von möglichst allen Mechanismen, nicht aufgrund der betreffenden Produkte.

Ein größer konzeptioneller Unterschied zum neuen Korrelationsformat ist, dass die bisherigen Korrelationsdaten mit dem Ziel erstellt werden, ein Inventar an Software-Artefakten zu modifizieren.
Es wird also immer ausschließlich von einem Artefakt ausgegangen, und explizit die Attribute definiert, die auf diesem gesetzt werden sollen.
Das neue Graphenformat verschiebt diesen Fokus zugunsten einer semantischen Modellierung, in der Bezüge zwischen Repräsentationen durch Kanten modelliert werden und die Effekte auf eine Quell-Repräsentation sich durch die Gefüge im Graphen ergeben.

Jeder Referenzfall stellt seine Einträge vor, aus denen die Herausforderungen und relevante Anforderungen erkannt werden.
Die Umsetzung dieser im neuen Modell zu Vergleichszwecken erfolgt in \autoref{sec:beispiele-fertige-implementierung}.

% FIXME-KKL: Zusammenhalten: Ich glaube der Kommentar hier bezieht sich darauf, dass im PDF zwischen dem Titel und der Klammer ein Abstand ist. Den habe ich schon versucht wegzubekommen, aber LaTex will das unbedingt so aufteilen.

\paragraph{JavaScript-Paket mit multiplen Namensvarianten}\label{par:reference-case-walletconnect} (s. \autoref{lst:correlation-generated-walletconnect-example})

Im Fall des Pakets \texttt{@web3-react/walletconnect}\footnote{\url{https://www.npmjs.com/package/@web3-react/walletconnect}} prägen sich trotz einer eindeutigen NPM-Paketidentität mehrere abweichende Bezeichner in realen Softwareinventaren aus.
Um all diese Varianten erfassen zu können, werden im alten Format mehrere nahezu identische Selektoren erstellt, was eine hohe Redundanz erzeugt.
Über diese Identifikation wird eine Zuordnung von \texttt{Additional CPE URIs} mit \texttt{cpe:/a:uniswap:web3-react\_walletconnect} vorgenommen.
Die Gründe für den Eintrag werden in dem Kommentarblock über diesem in einem \texttt{reason}-Feld dokumentiert, das auf NPM und die Herstellerseite verweist.

\textbf{Analyse:}
Die bisher genutzte Regex-Konstruktion \texttt{/(web|nodejs|npm)-module/} zur Typzuordnung ist fehleranfällig und nicht robust gegenüber neuen Typdefinitionen.
\hyperref[subsec:req-type-specific-matching]{A-06} verlangt, dass diese Erkennung zuverlässiger durchgeführt werden kann.
Ebenso demonstriert dieser Fall mit den \enquote{-*}-Suffixen die mögliche Übergeneralisierung der Einträge und zeigt die Notwendigkeit der Extraktion ökosystemspezifischer Attribute wie NPM-Namen oder Modulpfade.
Um die Mehrfachdefinition des Eintrags zu vereinheitlichen soll über \hyperref[subsec:req-multiple-attribute-values]{A-08} die Möglichkeit gegeben werden, in Attributselektoren Mehrfachwerte zu unterstützen.
Dies erlaubt eine kompaktere und konsistente Abbildung gleicher semantischer Bedeutung bei unterschiedlicher Schreibweise.

\paragraph{Java-Runtimes: Komplexe Versionstransformation}\label{par:reference-case-java-runtimes} (s. \autoref{lst:reference-case-java-runtimes})

Die generierten Korrelationseinträge für Java-Runtimes wie \enquote{Amazon Corretto} oder \enquote{Azul Zulu} basieren auf einem automatisierten Prozess, wie in \autoref{subsec:old-generated-correlation-data} beschrieben.
Sie wurden angelegt, um die uneinheitliche Strukturierung der \acrshort{cpe}-Versionen unter Kontrolle zu bringen, insbesondere den häufig genutzten \texttt{update}-Teil, der sonst in den meisten anderen \acrshort{cpe} oft ignoriert wird.
Einige Beispiele des Formats in den Versions- und Update-Attributen sind \texttt{21.0.6:*}, \texttt{23:*}, \texttt{*:update32}, \texttt{*:update\_32}, \texttt{1.6.0:update32\_b31} oder \texttt{1.6.0:update32\_b32}.
Im alten Format muss jede Kombination als separater Eintrag mit einem eigenständigen Selektor generiert werden, da pro Eintrag jeweils nur eine spezifische \acrshort{cpe}-Version möglich ist.

\textbf{Analyse:}
Vor allem mit der geplanten Erweiterung des generierten Datenanteils, macht die automatische Generierung solcher Daten eine systematische Strukturierung dieser Prozesse nötig (\hyperref[subsec:req-generated-data]{A-13}).
Die hier gezeigten Daten machen die Problematik mit der Erkennung von mehreren Eigenschaften über ein einziges Feld besonders gut sichtbar.
Es wird die Erkennung des Artefakts als Java Runtime, des Anbieters und der Version über ein einziges Feld \texttt{Id} mit einem regulären Ausdruck durchgeführt, was weder zuverlässig noch gut erweiterbar ist.
Zur Entkopplung dieser unterschiedlichen Fragestellungen ist die Anforderung \hyperref[subsec:req-type-specific-matching]{A-06} relevant, mit der eine Aufspaltung und typbasierte Verarbeitung von Attributen vorgesehen ist.

Im neuen Graphmodell wird in \hyperref[subsec:req-unique-product-representations]{A-05} verlangt, dass jede Repräsentation versionsunabhängig als genau \textit{ein} eindeutiger Knoten modelliert wird, der über einen Produkt-Knoten mit zentral gepflegten Attributen verbunden ist (\hyperref[subsec:req-product-concept]{A-04}).
Versionen und deren Transformationen sollen also auf dem Produkt-Knoten über Namensräume und Mappingregeln gepflegt werden, während die Erkennung eines Artefakts als Java Runtime mit einem konkreten Anbieter in einem Selektor auf einem Artefakt-Knoten vorgenommen wird.
Für die Erkennung der Versionen ist noch immer das Wildcard-System nötig (\hyperref[subsec:req-regex-support]{A-09}).

% FIXME-KKL: /zulu.*-(?:jre|jdk)-headless-11\.0\.10.*/i ist nicht korrekt: was wäre denn dann richtig? so steht es in den korrelationsdaten drin.
\begin{lstlisting}[style=yaml,caption={Java-Runtime-Korrelation mit Versionstransformation},label={lst:reference-case-java-runtimes},basicstyle=\ttfamily\scriptsize]
- affects:
    - Id: amazon-corretto-*/i
    - Id: /java-(?:\d\.?)+-amazon-corretto-.*/i
  append:
    Additional CPE URIs: cpe:/a:amazon:corretto, cpe:/a:oracle:jdk, cpe:/a:oracle:jre
- affects:
    - Id: /amazon-corretto-1\.1\.6.*9.*/i
    - Id: /amazon-corretto-java-1\.1\.6.*9.*/i
    - Id: /java-(?:\d\.?)+-amazon-corretto(?:-jdk)?-1\.1\.6.*9.*/i
  append:
    CPE URIs: cpe:/a:oracle:jre:1.1.6_009, cpe:/a:amazon:corretto

- affects:
    - Id: /zulu.*-(?:jre|jdk)-headless-.*/i
    - Id: /zulu.*-(?:jre|jdk)-.*/i
  append:
    Additional CPE URIs: cpe:/a:azul:zulu
- affects:
    - Id: /zulu.*-(?:jre|jdk)-headless-11\.0\.10.*/i
    - Id: /zulu.*-(?:jre|jdk)-11\.0\.10.*/i
  append:
    CPE URIs: cpe:/a:azul:zulu:11.0.10
- affects:
    - Id: /zulu.*-(?:jre|jdk)-headless-1\.8\.0.*[^0-9]282.*/i
    - Id: /zulu.*-(?:jre|jdk)-1\.8\.0.*[^0-9]282.*/i
  append:
    CPE URIs: cpe:/a:azul:zulu:8:update282
\end{lstlisting}

\paragraph{Redis: Kontextabhängige Produktdifferenzierung}\label{par:reference-case-redis} (s. \autoref{lst:correlation-order-depdendency-example})

Die Interpretation des Bezeichners \enquote{redis} innerhalb realer Softwareinventare ist ohne zusätzliche Attribute nicht eindeutig möglich.
Standardmäßig wird er also dem gleichnamigen Datenbankserver zugeordnet; liegt jedoch weiteres Attributwissen vor, kann auch das zugehörige Python-Clientmodul gemeint sein.
Da das alte Format keine explizite Modellierung der Beziehung zwischen Produkt und Repräsentation kennt, müssen zwei getrennte Einträge erzeugt werden, die miteinander interagieren.
Der generische Eintrag geht zunächst von der Server-Variante aus, während im spezifischeren Eintrag die generische Interpretation über \texttt{remove}- (und manchmal \texttt{ignores})-Felder explizit ausgeschlossen wird.
Dieses Verfahren ist fehleranfällig, insbesondere durch die Reihenfolgeabhängigkeit und Übersichtlichkeit bei wachsender Komplexität.

\textbf{Analyse:}
Das neue Modell führt in dem Graphen eine explizite Unterscheidung zwischen Produkt und Repräsentation ein (\hyperref[subsec:req-product-concept]{A-04}) und erlaubt damit eine getrennte Modellierung der beiden Repräsentationen durch individuelle Knotenpunkte.
Zwischen den Produkt-Knotenpunkten der Repräsentationen kann dann über das semantische Klassifikationssystem mit \enquote{is}/\enquote{is not}-Relationen spezifiziert werden, dass eine weitere angrenzende Repräsentation explizit oder explizit nicht zugeordnet werden soll (\hyperref[subsec:req-format-product-graph]{A-01}).
Mehrdeutigkeiten werden so nicht mehr über Auswertungsreihenfolgen und implizite Ausschlussmechanismen die das dreifache Aufführen einer \acrshort{cpe} verlangen, sondern nachvollziehbar (\hyperref[subsec:req-reason-format]{A-12}) durch explizite Kanten im Produktgraphen aufgelöst.

\paragraph{Windows 10: Betriebssystem mit mehreren Identifikatoren}\label{par:reference-case-windows} (s. \autoref{lst:reference-case-windows})

Die Modellierung von Windows 10 in den alten Korrelationsdaten stellt eine besondere Herausforderung durch die extreme Aufteilung der bekannten Repräsentationen des Betriebssystems nach Version, Architektur und weiteren Metriken dar.
So muss das Betriebssystem nicht nur je nach Ausprägung durch unterschiedliche \acrshortpl{cpe} repräsentiert werden, sondern auch zusätzliche Microsoft-Produkt-IDs und \acrshort{eol}-Ids.
Die bisherige Lösung erfordert die Wiederholung von Attributkombinationen zwischen unterschiedlichen Einträgen, als auch innerhalb einzelner Einträge für verschiedene \texttt{Id}-Ausprägungen wie sie in den Quelldaten auftreten können.
Diese Redundanz ist bei den meisten Fällen durch die geringe Anzahl an Wiederholungen ertragbar, jedoch wird besonders bei der Abbildung der großen Menge an hierarchisch angeordneten Windows-Versionen (Windows 10 → 21H2 → Architektur) die Beziehungen, die nur implizit angelegt und erkannt werden können, problematisch.

\textbf{Analyse:}
Die Unfähigkeit zur kompakten Darstellung mehrerer Attributwerte innerhalb eines Eintrags wird durch \hyperref[subsec:req-multiple-attribute-values]{A-08} angesprochen.
Die Kernproblematik liegt jedoch in einem fehlenden Mechanismus zum Teilen von Attributen zwischen Artefakt-Selektoren getrennter Einträge, welche in \hyperref[subsec:req-selektor-inheritance]{A-07} durch Vererbungsregeln adressiert wird.
Durch die hierarchischen Knotenstrukturen im neuen Modell soll ein allgemeiner Windows-10-Produktknoten generische Eigenschaften definieren, von dem spezifische Versionen (z.\ B. 21H2) erben, welche wiederum architekturspezifische Knoten (x64/x86) als Kinder enthalten können.
An jeden Artefakt-Knoten in der Hierarchie wird ein Produkt-Knoten angehängt, über den weitere versionsabhängige Repräsentationen (\acrshort{cpe}, Microsoft-Produkt-ID, \acrshort{eol}-Id) angehängt werden.
Redundanzen entfallen durch die Vererbung gemeinsamer Attribute entlang der Knotenhierarchie und das Aufführen mehrerer alternativer Werte in einem Selektor-Attribut.

\begin{lstlisting}[style=yaml,caption={Windows-Korrelation mit mehreren Identifikatoren},label={lst:reference-case-windows},basicstyle=\ttfamily\scriptsize]
- affects:
    - Id: Windows 10*
      Type: operating system
    - Id: Microsoft Windows 10*
      Type: operating system
  remove:
    Additional CPE URIs: cpe:/o:microsoft:windows
  append:
    Inapplicable CPE URIs: cpe:/a:windows:media_player, cpe:/o:mircorsoft:windows, cpe:/o:microsoft:windows
    Additional CPE URIs: cpe:/o:microsoft:windows_10
    EOL Id: windows

# reason: https://learn.microsoft.com/de-de/windows/release-health/release-information
#         11931 --> Version 21H2 (OS build 19044) / Windows 10 Version 21H2 for x64-based Systems
- affects:
    - Id: Microsoft Windows 10*
      Version: 10.0.19044*
      Type: operating system
      Architecture: "*64*"
    - Id: Windows 10*
      Version: 10.0.19044*
      Type: operating system
      Architecture: "*64*"
  append:
    MS Product ID: "11931"

# reason: 11929 --> Version 21H2 (OS build 19044) / Windows 10 Version 21H2 for 32-bit Systems
- affects:
    - Id: Microsoft Windows 10*
      Version: 10.0.19044*
      Type: operating system
      Architecture: "*32*"
    - Id: Windows 10*
      Version: 10.0.19044*
      Type: operating system
      Architecture: "*32*"
  append:
    MS Product ID: "11929"

# reason: 11929 --> Version 21H2 (OS build 19044) / Windows 10 Version 21H2 for 32-bit Systems
- affects:
    - Id: Microsoft Windows 10*
      Version: 10.0.19044*
      Type: operating system
  append:
    Additional CPE URIs: cpe:/o:microsoft:windows_10_21h2, cpe:/o:microsoft:windows_10:21h2
\end{lstlisting}
