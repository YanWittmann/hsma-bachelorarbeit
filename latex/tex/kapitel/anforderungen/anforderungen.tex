\section{Anforderungen an das neue Korrelationssystem}\label{sec:requirements}

% “4.3. Data Types in Falcon” ([Cheramangalath et al., 2016, p. 6](zotero://select/library/items/4DG8G9J3)) ([pdf](zotero://open-pdf/library/items/QNYKYEH4?page=6&annotation=C7MKFAUG))

Basierend auf der Analyse der bestehenden Herausforderungen und weiteren Rahmenbedingungen werden die Anforderungen an das neue Korrelationssystem abgeleitet.

\subsection{A-01: Graphenbasiertes Datenmodell mit expliziten Abhängigkeitsdeklarationen}\label{subsec:req-format-product-graph}

Das Kernmodell des neuen Systems muss als gerichteter Graph implementiert werden, wobei Knoten typisierten Produktrepräsentationen (Artefakte, \acrshort{cpe}, \acrshort{eol}-Id, MS Product ID) repräsentieren und Kanten Relationen mit unterschiedlichen Typen zwischen diesen darstellen.
Kantentypen umfassen mindestens \enquote{is} (represented by, positive Zugehörigkeit) und \enquote{is not} (not represented by, expliziter Ausschluss).
Der Typ des Zielknotens einer Relation entscheidet, wie die Art der Relation interpretiert werden soll.

Jede Repräsentation eines Produkts darf sich in nur einen Knoten ausprägen, Mehrfachreferenzen auf dieselbe Repräsentation müssen durch Kanten auf einen einheitlichen Knoten dargestellt werden.
Dies löst die Reihenfolgenabhängigkeit (\hyperref[subsec:c-10-order-dependency]{C-10}), indem die Abhängigkeiten zwischen Knoten explizit deklariert werden.
Diese Deklaration soll eine topologische Sortierung während der Verarbeitung erzwingen und implizite Dateireihenfolgeabhängigkeiten vollständig entfernen.

Um die programmatische Erfassung eines Knotenpunktes zu ermöglichen, soll jeder Knotenpunkt einen textuellen Identifikator besitzen, der in Kombination mit dem Knotentyp im Graph eindeutig ist.
So kann über die Typ-Id-Kombination jeder Knoten im Graph angesprochen werden.

Der Graph ist gerichtet, damit über die Kantenrichtung die \enquote{is}- und \enquote{is not}-Beziehungen klar ausgewertet werden können.
Falls es nötig ist, kann eine identische, umgekehrte Kante dazu verwendet werden, um die Richtung der Kante auf beide Seiten zu erweitern.
In den Fall, dass der Bedarf besteht herauszufinden, welche weiteren Repräsentationen eine Produktrepräsentation noch haben kann, muss damit also bei dem als \enquote{sich selbst} identifizierten Knoten eine Durchquerung des Graphen unter Berücksichtigung der Kantenrichtung- und Typen gestartet werden, bis alle erreichbaren Knoten gefunden und ausgewertet wurden.

Mit diesem System sollen jedwede Produkte und Repräsentationen modelliert werden können.
Dies löst die Herausforderung \hyperref[subsec:c-06-falle-ohne-aktion-konnen-nicht-dokumentiert-werden]{C-06}, dass keine Identifikationen ohne bekannte Relationen modelliert werden können, da in diesem System dennoch ein Knotenpunkt für die Identifikation erstellt werden kann, ohne eine Relation zu einem weiteren Knoten erstellen zu müssen.

\subsection{A-02: Produkt-Konzept}\label{subsec:req-product-concept}

Wichtig ist die Unterscheidung zwischen einem \textit{Produkt}, und der \textit{Repräsentation eines Produkts} (s.\ \autoref{subsec:produkte-vs-reprasentation}).
Diese sollen sich zwar im Graphen beide als Knotenpunkte darstellen, jedoch muss zwischen ihren Typen unterschieden werden.

So könnte eine Repräsentation mit einer Kante auf einen Produkt-Knotenpunkt verweisen und von diesem eine Kante auf eine weitere Repräsentation des Produkts um anzugeben, dass diese Repräsentationen das selbe Produkt darstellen.

Produktknoten sollen in der Lage sein, Metainformationen über die zugehörigen Repräsentationen abzulegen.
Dazu gehören vor allem Informationen über die Versionsräume und wie zwischen diesen umgewandelt werden kann (\hyperref[C-13]{subsec:c-13-attribut-abbildung-unzureichend}).

\subsection{A-03: Normalisierte Typidentifikation und typspezifische Attribute für Software-Artefakte}\label{subsec:req-type-specific-matching}

Um die Herausforderung \hyperref[subsec:c-02-uneindeutige-artefakt-typinformation]{C-02} der multiplen Ausprägungen und inkonsistenten Typinformationen zu lösen muss eine Typinferenz-Logik den Artefakttypen konsistent aus multiplen Quellen aufbereiten.
In \autoref{subsec:erkennung-typspezifische-artefakte} wurden die häufigsten Muster für Typinformationen analysiert.
Die Ausgabe der Typinterferenz soll diese verwenden, um normalisierte Ökosystembezeichner (z.\ B. \texttt{java-module}, \texttt{npm-package}, \texttt{python-module}) und weitere Typspezifische Attribute als Vergleichswerte verfügbar machen.

Basierend auf dem erkannten Ökosystem/Typ eines Artefakts werden dann weitere Extraktoren auf die Artefakt-Metadaten angewendet.
Diese müssen weitere Attribute als First-Class Matching-Kriterien auf den Artefakt-Selektoren zur Verfügung stellen, wie etwa Maven-Koordinaten (Group Id, Artifact Id) für Java, Paketnamen für NPM, Distributionskennung für Linux-Pakete, Java Runtime-Provider, und viele mehr.
Wildcard-Selektor wie bisher häufig in der \texttt{Id} (\hyperref[subsec:c-01-unspezifische-identifikation-von-artefakten]{C-01}) können so durch exakte, ökosystemspezifische Attributvergleiche ersetzt werden.

Falls kein bekannter Typ konkreter erkannt werden konnte, muss ein Artefakt dennoch mit den Basisattributen zum Abgleich zur Verfügung gestellt werden.

\subsection{A-04: Vererbung von Artefaktmerkmalen}\label{subsec:req-selektor-inheritance}

Um zu vermeiden, dass Artefakt-Selektoren wie in \hyperref[subsec:c-03-duplizierte-artefakt-selektoren]{C-03} für ähnliche Identifikationen wiederholt werden müssen, soll es ein Vererbungssystem an Artefakt-Selektoren geben.
Ein Basis-Selektor definiert generische Selektorattribute (z.\ B.\ für alle Microsoft Windows-Varianten), während abgeleitete Selektoren spezifische Erweiterungen (z.\ B.\ Architektur) hinzufügen, wobei lokale Attributdefinitionen geerbte überschreiben.
Dies erlaubt das Teilen von Basisattributen zwischen mehreren spezifischeren Ausprägungen eines Artefakts, ohne die Basisattribute bei jeder Ausprägung wiederholen zu müssen und vermeidet damit zu pflegende Redundanz.
So muss ebenfalls nur der Basis-Selektor angepasst werden, wenn neue Attribute dazukommen oder vorhandene bearbeitet werden sollen.

\subsection{A-05: Auflistung mehrerer Werte pro Attribut}\label{subsec:req-multiple-attribute-values}

In \hyperref[subsec:c-03-duplizierte-artefakt-selektoren]{C-03} wurde die Herausforderungen der mehreren Werte pro Attribut aufgeführt.
Um dieses zu lösen, soll im neuen Modell ein Abgleich von mehreren Optionen möglich sein, bei der nur ein Attributwert übereinstimmen muss (oder-Verknüpfung).

\subsection{A-06: Unterstützung regulärer Ausdrücke}\label{subsec:req-regex-support}

Auch wenn im neuen Format über \hyperref[subsec:req-type-specific-matching]{A-03} die Anzahl der Wildcard-Selektor drastisch verringert werden, soll dennoch das in \autoref{sec:current-correlation-format} beschriebene inkrementelle Wildcard-System weiterhin bestehen bleiben.
% FIXME-KKL: Option: mit Mustern und Variablen arbeiten. Alternativen erwähnen.: Was wären Alternativen? Ich habe es momentan nicht implementiert, Variablen zu ermöglichen, soll dass dann eine Erweiterung sein?

\subsection{A-07: Manuelle Modifikation des Graphen}\label{subsec:req-manual-format-modification}

Da in dem neuen Korrelationssystem noch immer manuelle Korrekturen an der automatischen Erkennung von \acrshort{cpe} gemacht werden können sollen, muss ein neues \acrshort{yaml}-basiertes Format entwickelt werden, welches einen solchen Produkt-Graphen modifizieren kann.
Die durchführbaren Modifikationen am Graphen müssen mindestens die Erstellung, Modifikation und Entfernung von Knotenpunkten jedes Types, und die Erstellung, Modifikation und Entfernung von Kanten zwischen Knoten beinhalten.
Technisch sollte dieses Modifikationsformat auf eine Weise modelliert werden, dass jegliche nötigen Modifikationen am Graphen ausschließlich im Format lösbar sind, um als allgemeine Schnittstelle für Graphmodifikationen zu dienen.
Auf diese Weise können auch andere Prozesse dieselbe Logik verwenden, um ihre Änderungen durchzuführen, mit den gleichen Garantien wie der manuelle Prozess.

Da die Kombination aus Typ und Id jedes Knotens im Graphen eindeutig ist (s. \hyperref[subsec:req-format-product-graph]{A-01}), sollte auch das Modifikationsformat diese Weise verwenden, Knotenpunkte für die Modifikation zu selektieren.

Das neue Korrelationssystem als Graph zu modellieren führt viele Komplexitäten ein, die für den Nutzer im manuellen Modifikationsformat abstrahiert werden müssen, um noch immer so effektiv wie im alten Format arbeiten zu können.
Hierfür muss ein Format iterativ entworfen und mit aktuellen Nutzern des Formats getestet und abgesprochen werden.
Das Ziel ist es, es so einfach wie möglich zu machen, den Graphen zu modifizieren, aber trotzdem noch jedes Attribut modifizieren zu können.

Da das alte Format oft Schwierigkeiten hatte, aus dem Datenmodell wieder die ursprünglichen Einträge in den \acrshort{yaml}-Dateien zu finden (s.\ \hyperref[subsec:c-11-finding-yaml-entries]{C-11}), soll in dem neuen \acrshort{yaml}-Modifikationsformat das Auffinden von Einträgen deterministisch gestaltet werden.
Wenn zur Selektion der Knotenpunkte im \acrshort{yaml} der Typ und die Id verwendet wird, dann kann diese auch umgekehrt wieder aus dem Graphen ausgelesen und die zugehörige Zeile im \acrshort{yaml} gefunden werden.

\subsection{A-08: Maschinenlesbare Entscheidungsdokumentation}\label{subsec:req-reason-format}

Um Daten wie Projektreferenzen, Beschreibungen und Begründung für Entscheidungen im Gegensatz zum alten Korrelationssystem (\hyperref[subsec:c-05-reason-not-good-enough]{C-05}) maschinenlesbar zu machen, müssen die bisherigen Kommentarfelder als strukturierte Objekte mit standardisierten Feldern modelliert werden.
Dies soll es in automatisierten Systemen oder Nutzeroberflächen möglich machen, Metadaten über die Produktidentifikationen nützlicher anzuzeigen und auswerten zu können.

Da das neue Korrelationssystem mit einem Graphen modelliert wird, gibt es zwei Stellen, an denen diese Dokumentation angebracht werden kann und muss.
So wird der frühere \texttt{\# reason:}-Kommentar mit der Begründung der gegenseitigen Anwendbarkeit von Produkten nun auf den Kanten modelliert, und die weiteren Metadaten eines Produkts wie die Beschreibung oder Web-Referenzen auf den Knotenpuntken.

\subsection{A-09: Generierungsframework für dynamische Daten}\label{subsec:req-generated-data}

Im aktuellen Korrelationssystem liegen die generierten \acrshort{yaml}-Dateien gleichwertig neben den manuell gepflegten Daten (\hyperref[subsec:c-08-generated-correlation-data]{C-08}).
Da ihre Existenz nicht von Beginn an geplant war, wurden das Selektor-System und die weiteren Matching-Regeln nicht für die besonderen Anforderungen die diese Quellen mit sich bringen ausgelegt.
Im neuen System soll der gesamte Graph darauf aufbauen, dass er zunächst mit generierten Daten befüllt wird und basierend darauf die manuellen Modifikationen angewendet werden.

Mit den Datensätzen der \metaeffektsp sollen bereits alle Knotenpunkte für \acrshortpl{cpe}, MS Product Ids und \acrshort{eol}-Ids erzeugt werden, die bekannt sind, manche sollen bereits Kanten zwischen den Knoten erzeugen.
Zudem sollen weitere Datenquellen wie die bereits vorhandenen Generatoren für NPM-Pakete zu \acrshortpl{cpe} und die für die vielen Java Runtimes, aber auch neue wie purl2cpe von scanoss\footnote{\url{https://github.com/scanoss/purl2cpe}} dazu beitragen, den Graphen bereits vorzufüllen und die darauf folgende Arbeit zu erleichtern.
% FIXME-KKL: Siehe Erweiterungsvorschlag oben: welcher vorschlag?

Um den fertigen auslieferbaren Graphen zu erhalten, soll zunächst der generierte Anteil des Graphen erstellt werden und dann auf diesem mit den manuellen Modifikationen aufgebaut werden.
Durch diese Trennung in mehrere \enquote{Contributors} am Graphen kann \hyperref[subsec:c-09-sharing-of-public-data]{C-09} einfach abgebildet werden, indem der Datensatz vor der Anwendung des manuellen Schritts ausgeliefert wird.

\subsection{A-10: Qualitätsmetriken des Graphen}\label{subsec:req-graph-inner-consistency}

%- Innere Konsistenz: Zu jeder Repräsentation, die durch das Modell identifiziert werden soll, darf maximal eine
%  eindeutige Identifikation stattfinden und es darf keine losen Knoten geben
%- Datensatz erklärt sich selbst: Zu jedem Eintrag und jeder Verbindung muss es eine Begründung jeglicher Art geben
%- Ein sich selbst prüfender Datensatz: Nachdem ein Datensatz manuell geprüft wurde, werden alle Identifikationen in
%  einem separaten Datensatz abgelegt, um bei zukünftigen Änderungen automatisch geprüft werden zu können. So soll
%  gegeben sein, dass die Identifikation eines bekannten Produktes nicht einfach so ändern kann, ohne, dass man es
%  mitbekommen würde.

Um verifizieren zu können, dass der Graph sinnhaft strukturiert und in sich selbst konsistent ist, müssen einige Prüfungen darauf ablaufen können.
Diese prüfen mindestens die folgenden Eigenschaften:

\begin{itemize}
    \item Zu jeder Repräsentation eines Produkts in dem Graphen darf maximal genau eine eindeutige Identifikation stattfinden.
    Dies bedeutet, dass jeder Knoten exakt eine Repräsentation vollständig modellieren sollte und diese nicht noch einmal von einem anderen Knoten partiell abgedeckt sein darf.
    Diese Metrik muss sowohl in dem Matching-Algorithmus, als auch als Modellierungsvorschrift umgesetzt werden.
    \item Eine Erweiterung zu dieser Metrik ist, dass pro Artefakt im Graphen nur ein einziger Produktknoten über eine \enquote{is}-Beziehung erreicht werden darf.
    Es muss also eine Prüfung auf die Erreichbarkeit und Überlappungen von Produktknoten von Artefaktknoten aus durchgeführt werden.
    % FIXME-KKL: das hier hat halt das Problem, dass wir den Graphen mit allen CPE, etc. füllen wollen und nicht immer ein zugehöriges Produkt erzeugt werden kann.
    \item Es darf keine losen Knoten geben, jede Repräsentation muss mindestens zu einem Produkt-Knotenpunkt verbunden sein um sicherzustellen, dass mit einer Identifikation auch ein Informationsgewinn stattfinden kann.
    \item Um einen sich selbst erklärenden Datensatz zu erhalten, müssen die Metadaten jedes Knotenpunktes aufgefüllt sein und die Kanten eine Begründung enthalten, warum sie so existieren.
    \item Der Graph soll sich selbst prüfen können: Nachdem ein Software-Inventar manuell geprüft wurde, sollen alle Quell-Artefakte mit ihren Identifikationen in einem separaten Datensatz abgelegt werden, um diese bei zukünftigen Änderungen am Graphen automatisch prüfen zu können.
    So kann die Identifikation eines bekannten Produktes sich nicht einfach so ändern, ohne, dass man es mitbekommen würde.
    \item Eine Prüfung auf zirkuläre Vererbungs-Referenzen soll sicherstellen, dass es immer einen spezifischeren Knoten gibt und keine Schleife entsteht.
\end{itemize}

\subsection{A-11: Organisation der \acrshort{yaml}-Dateien}\label{subsec:req-yaml-file-organization}

Die manuell gepflegten Korrelationsdateien werden noch immer im Dateisystem abgelegt werden.
Hierbei müssen jedoch die Herausforderungen der tausenden Zeilen langen \acrshort{yaml}-Dateien adressiert werden (\hyperref[subsec:c-04-groe-und-unubersichtliche-yaml-dateien]{C-04}).
Es soll für die neuen Dateien mindestens eine Auftrennung in Verzeichnissen nach Typ und/oder Ökosystem geben, und weiterführend können die einzelnen Hersteller- oder Produktgruppen jeweils in getrennten Dateien aufgeführt werden.

\subsection{A-12: Überführung von Einträgen aus dem alten Korrelationsdatensatz}\label{subsec:req-current-dataset-conversion}

Um die Transition zum neuen Korrelationssystem so einfach wie möglich zu machen, sollen die beiden Datensätze zunächst parallel weiter geführt und auf Inventare angewendet werden.
In der Transitionsperiode sollen dann die Einträge aus dem alten Format in das neue konvertiert und damit der alte langsam ersetzt werden.

Dazu muss sichergestellt werden, dass der neue Graph mindestens die Fähigkeiten des alten Formats abdeckt und durch seine weiteren Eigenschaften die Identifikation sogar verbessert.

\subsection{A-13: Performance des neuen Korrelationsformats}\label{subsec:req-correlation-format-performance}

Das alte Korrelationssystem wurde bei unterschiedlichen Inventar-Größen mit dem aktuellen Datensatz mit 6400 Korrelationseinträgen getestet.
Die Zeiten werden in \autoref{tab:old-correlation-performance} angegeben.

Aus diesen Messwerten sind zwei Phasen erkennbar.
Zunächst läuft eine konstante Startzeit (bei dem getesteten Datensatz etwa 0,8-0,9 Sekunden), die unabhängig von der Anzahl der Artefakte im Inventar auftritt, in welcher die Korrelationseinträge aus dem \acrshort{yaml} ausgelesen werden.
Danach steigt die Laufzeit annähernd linear mit der Anzahl der Artefakte an, der Zeitbedarf pro zusätzlichem Artefakt ist also relativ konstant.

\begin{table}[h!]
    \centering
    \begin{tabular}{l r r r r}
        \toprule
        \textbf{Artefakte} & \textbf{Ø [s]} & \textbf{Min. [s]} & \textbf{Max. [s]} \\
        \midrule
        0                  & 0,806          & 0,435             & 1,668             \\
        1                  & 0,897          & 0,453             & 1,919             \\
        500                & 2,033          & 1,504             & 3,277             \\
        1000               & 3,256          & 2,531             & 4,612             \\
        2000               & 5,920          & 4,543             & 6,979             \\
        3000               & 8,804          & 7,425             & 11,800            \\
        4000               & 9,433          & 8,780             & 10,338            \\
        5000               & 12,162         & 10,929            & 13,596            \\
        6000               & 13,609         & 12,461            & 15,381            \\
        \bottomrule
    \end{tabular}
    \caption{Gemessene Laufzeiten des alten Korrelationsformats bei unterschiedlichen Artefakt-Anzahlen.}
    \label{tab:old-correlation-performance}
\end{table}

Da der Korrelationsschritt bisher bei weitem nicht der zeitintensivste Schritt in der Schwachstellenanalyse ist, ist die Optimierung nie eine kritische Anforderung gewesen.
Das neue Korrelationsmodell sollte diese Zeiten jedoch bei vergleichbaren Datenmengen dennoch nicht um den Faktor von 2 überschreiten.

% \begin{table}[h!]
%     \centering
%     \label{tab:new-correlation-performance}
%     \begin{tabular}{l r r r}
%         \toprule
%         \textbf{Artefakte} & \textbf{Ø [s]} & \textbf{Min. [s]} & \textbf{Max. [s]} \\
%         \midrule
%         0                  & 0,000          & 0,000             & 0,000             \\
%         1                  & 0,034          & 0,001             & 0,134             \\
%         500                & 2,303          & 2,078             & 2,950             \\
%         1000               & 4,249          & 4,022             & 4,790             \\
%         2000               & 8,582          & 8,109             & 9,658             \\
%         3000               & 12,639         & 12,156            & 13,730            \\
%         4000               & 16,702         & 16,113            & 17,446            \\
%         5000               & 20,693         & 20,179            & 21,175            \\
%         6000               & 24,601         & 24,093            & 25,503            \\
%         \bottomrule
%     \end{tabular}
%     \caption{Laufzeiten des neuen Korrelationsformats bei unterschiedlichen Artefakt-Anzahlen.}
% \end{table}

\subsection{A-14: Auslieferung des Datensatzes}\label{subsec:req-correlation-data-delivery}

Der Aufwand, den Korrelationsdatensatz zu verteilen, muss vergleichbar mit dem alten Korrelationssystem sein.
Bisher hat es ausgereicht, einen Ordner an \acrshort{yaml}-Dateien zu verteilen.
Es darf also keine dedizierte Ausführungsumgebung einer Datenbank vorausgesetzt werden, der gesamte Ablauf muss innerhalb des Java-Prozesses stattfinden, in dem auch das alte Format verwendet wird.

\subsection{A-15: Zu verwendende Technologien}\label{subsec:req-lang-java}

Da die bestehende Software zum Schwachstellenmanagement der \metaeffektsp in Java geschrieben ist, muss auch die Implementierung des neuen Korrelationssystems in Java geschehen.
Zum Abspeichern des Graphen kommen die Bibliotheken infrage, die bereits auf dem Classpath des Projekts sind, um weitere Abhängigkeiten zu vermeiden.
Eine Datenbank würde sich zur performanten Abfrage und Transformation eignen.
