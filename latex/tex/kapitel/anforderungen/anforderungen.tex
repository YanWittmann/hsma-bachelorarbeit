\section{Anforderungen an das neue Korrelationssystem}\label{sec:requirements}

Basierend auf der Analyse der bestehenden Herausforderungen und weiteren Rahmenbedingungen werden die Anforderungen an das neue Korrelationssystem abgeleitet.

\subsection{A-GRAPH-MODEL: Graphstruktur als Datenmodell}\label{subsec:req-format-product-graph}

\textbf{Ziel:}
Das System muss ein gerichtetes Graphmodell verwenden, in dem Knoten typisierte Produktrepräsentationen (z.\ B. Artefakte, \acrshort{cpe}, \acrshort{eol}-Id, MS Product ID) abbilden und Kanten explizite Relationen zwischen diesen darstellen.
Mindestens die Relationstypen \enquote{is} (positive Zugehörigkeit) und \enquote{is not} (expliziter Ausschluss) sind zu unterstützen.

Die Richtung der Kanten steuert die semantische Auswertung.
Bidirektionale Verbindungen werden durch zusätzliche Kanten in umgekehrter Richtung modelliert.
Um alternative Repräsentationen eines Produkts zu bestimmen, muss eine gerichtete Traversierung aller \enquote{is}-Kanten vom Ausgangsknoten durchgeführt und ausgewertet werden.

\textbf{Begründung:}
Die Verwendung eines Graphmodells zur Abbildung der Produktbeziehungen und deren Auswertung ist eine Rahmenbedingung der Arbeit.

\textbf{Fit-Kriterium:}
Das System verwendet ein gerichtetes Graphmodell mit Knoten für typisierte Produktrepräsentationen und Kanten für Relationstypen \enquote{is} und \enquote{is not}.
Traversierungen der Kanten sind möglich und können ausgewertet werden.

\subsection{A-NODE-IDENTIFICATION: Knotenidentifikation}\label{subsec:req-node-id-type}

\textbf{Ziel:}
Jeder Knotenpunkt besitzt einen textuellen Identifikator, der in Kombination mit dem Knotentyp im Graph eindeutig ist.

\textbf{Begründung:}
Dies ermöglicht eine programmatische Erfassung eines Knotens ausschließlich über die Typ-Id-Kombination.
Einen Eintrag im alten Korrelationssystem in den Eingabedaten zu identifizieren war eine nicht vollständig lösbare Herausforderung, durch dieses Format lässt sich ein Knoten im neuen Modell in jeder Form identifizieren (\hyperref[subsec:c-11-finding-yaml-entries]{C-11}).

\textbf{Fit-Kriterium:}
Jeder Knoten besitzt eine Typ-Id-Kombination, die im Graph eindeutig ist und den Zugriff sowie die Manipulation des Knotens erlaubt.

\subsection{A-NODES-WITHOUT-IDENTIFICATIONS: Modellierung isolierter Identifikationen}\label{subsec:req-nodes-without-identification}

\textbf{Ziel:}
Das System muss ein Gefüge an Knoten erlauben, in dem Repräsentationen ohne Kanten zu weiteren Repräsentationen modelliert werden können, um die Dokumentation von Produkten ohne Abbildungen zu ermöglichen.

\textbf{Begründung:}
So können isolierte Identifikationen abgebildet werden, was \hyperref[subsec:c-06-falle-ohne-aktion-konnen-nicht-dokumentiert-werden]{C-06} löst.

\textbf{Fit-Kriterium:}
Repräsentationen können ohne Abbildung zu weiteren Repräsentationen im Graph vorhanden sein.

\subsection{A-PRODUCT-CONCEPT: Produkt-Konzept}\label{subsec:req-product-concept}

\textbf{Ziel:}
Die Unterscheidung zwischen einem \enquote{Produkt} und der \enquote{Repräsentation eines Produkts} (s.\ \autoref{subsec:produkte-vs-reprasentation}).
Im Graphen werden beide als Knotenpunkte dargestellt, jedoch muss zwischen ihren Typen bei der Auswertung und Modellierung unterschieden werden.

Produktknoten sind als Zentrale Knotenpunkte zwischen Repräsentationen in der Lage, Metainformationen über das umliegende Produkt-Ökosystem abzulegen.
Dazu gehören vor allem Informationen über die Versionsräume und wie zwischen diesen umgewandelt werden kann (\hyperref[C-13]{subsec:c-13-attribut-abbildung-unzureichend}).
Produkte können mit einer \enquote{is not}-Kante verbunden sein.
In diesem Fall müssen alle von dem Produkt betroffenen Knoten ebenfalls als eine negative Zugehörigkeit behandelt werden.

\textbf{Begründung:}
Die Trennung in Repräsentationen und Produkte ermöglicht eine bewusste Modellierung eines Produkt-Ökosystems, deren Beziehungen und eine Abbildung von Produktvarianten und Versionen.

\textbf{Fit-Kriterium:}
Im Graph existieren unterschiedliche Knotentypen für Produkte und Repräsentationen.
Produktknoten enthalten Metainformationen zu umliegenden Repräsentationen und deren Versionsräumen.
Repräsentationen sind über Kanten mit dem Produkt verbunden.

\subsection{A-UNIQUE-PRODUCT-REPRESENTATION: Eindeutigkeit von Repräsentationen}\label{subsec:req-unique-product-representations}

\textbf{Ziel:}
Jedes Produkt und jede Repräsentation eines Produkts darf sich in nur einem Knoten ausprägen.
Versionierte Ausprägungen sollen vermieden werden, wann auch immer möglich.
Alle Repräsentationen eines Produkts müssen über Kanten mit dem gemeinsamen Produktknoten verbunden werden.
Bei einer Negatividentifikation wird kein weiterer Knoten einer Repräsentation erzeugt, sondern der bereits vorhandene mit einer \enquote{is not}-Kante wiederverwendet.

\textbf{Begründung:}
Durch diese Limitierung an Knoten kann sichergestellt werden, dass der Abgleich einer Repräsentation von außen sich immer mit nur genau einen Knoten im Graphen identifizieren lässt.
Zudem wird die Reihenfolgenabhängigkeit (\hyperref[subsec:c-10-order-dependency]{C-10}) durch explizite Abhängigkeiten gelöst.

\textbf{Fit-Kriterium:}
Im Graph existiert pro Repräsentation exakt ein Knoten, der über Kanten mit anderen Repräsentationen desselben Produkts verbunden ist.

\subsection{A-ARTIFACT-TYPES-IDENTIFICATION: Normalisierte Typidentifikation und typspezifische Attribute für Software-Artefakte}\label{subsec:req-type-specific-matching}

\textbf{Ziel:}
Im Bezug auf Artefakte als Datenmodell muss eine Typinferenz-Logik existieren, die Artefakttypen konsistent aus der Kombination von mehreren Attributen identifiziert und auf normalisierte Ökosystembezeichner (z.B. \texttt{java-module}, \texttt{npm-package}, \texttt{python-module}) abbildet.
Je nach erkannten Typ müssen weitere, typspezifische Extraktoren bereitgestellt werden, die relevante Attribute (wie Maven-Koordinaten für Java oder Paketnamen für NPM) als First-Class Matching-Kriterien in den Artefakt-Selektoren exponieren.
Für nicht erkennbare Typen muss auf Basisattribute zurückgefallen werden.

\textbf{Begründung:}
Diese Anforderung löst die Probleme inkonsistenter Typinformationen aus \hyperref[subsec:c-02-uneindeutige-artefakt-typinformation]{C-02} und reduziert unspezifische Identifikationen nach \hyperref[subsec:c-01-unspezifische-identifikation-von-artefakten]{C-01} durch ökosystemspezifische Attribut-Gleicheitsvergleiche.

\textbf{Fit-Kriterium:}
Die in \autoref{subsec:erkennung-typspezifische-artefakte} aufgeführten Artefakt-Typen müssen mindestens unterstützt werden.
Dazu gehört die Erkennung des Typs, aber auch die Extraktion von den aufgeführten relevanten Attributen.
Diese müssen für den Abgleich verfügbar gemacht werden.
Artefakte unbekannter Typen werden mit Basisattributen bereitgestellt.

\subsection{A-ARTIFACT-INHERITANCE: Vererbung von Artefaktmerkmalen}\label{subsec:req-selektor-inheritance}

\textbf{Ziel:}
Ein Vererbungssystem für Artefakt-Selektoren muss existieren, bei dem mit Basis-Selektoren generische Attribute (z.\ B.\ für alle Microsoft Windows-Varianten) definiert werden können und abgeleitete Selektoren spezifische Erweiterungen (z.\ B.\ Architektur) hinzufügen.
Bei der Auswertung eines Artefakts über die Vererbungshierarchie müssen alle generischen Selektoren ebenfalls zu dem Artefakt zutreffen.
Es wird ausschließlich der spezifischste Knoten in der Kette gewählt.

\textbf{Begründung:}
Dies erlaubt das Teilen von Basisattributen zwischen mehreren spezifischeren Ausprägungen eines Artefakts, ohne die Basisattribute bei jeder Ausprägung wiederholen zu müssen und vermeidet damit zu pflegende Redundanz (\hyperref[subsec:c-03-duplizierte-artefakt-selektoren]{C-03}).

\textbf{Fit-Kriterium:}
Die Logik der Vererbungshierarchie von Artefakt-Selektoren muss korrekt implementiert sein.
Basis-Knoten definieren dabei generische Selektoren, während abgeleitete Knoten Selektoren mit ergänzenden Attributen führen können.
Bei der Auswertung eines Artefakts müssen die in der Anforderung geführten Regeln zutreffen.

\subsection{A-MULTIPLE-VALUES-PER-ATTRIBUTE: Auflistung mehrerer Werte pro Attribut}\label{subsec:req-multiple-attribute-values}

\textbf{Ziel:}
Artefakt-Selektoren müssen je Attribut mehrere Werte als ODER-Verknüpfung unterstützen.
Der Abgleich mit den realen Werten gilt als erfolgreich, wenn mindestens ein Wert übereinstimmt.

\textbf{Begründung:}
Lösung der Herausforderung \hyperref[subsec:c-03-duplizierte-artefakt-selektoren]{C-03} durch Vermeidung redundanter duplizierter Selektoren bei mehreren gültigen Werten für ein Attribut.

\textbf{Fit-Kriterium:}
Jedes Attribut der Artefakt-Selektoren auf das geprüft werden kann akzeptiert Wertelisten.
Der Abgleich ist erfolgreich, wenn ein Listenelement mit dem Zielartefakt übereinstimmt.

\subsection{A-REGULAR-EXPRESSIONS: Unterstützung regulärer Ausdrücke}\label{subsec:req-regex-support}

\textbf{Ziel:}
Artefakt-Selektoren sollen über das bestehende inkrementelle Wildcard-System, wie in \autoref{sec:current-correlation-format} beschrieben, weiterhin bestehen bleiben.
Mit dem System können mehrere Stufen von Text-Vergleichen bis hin zu vollen regulären Ausdrücken mit Flags unterstützt werden.
Das zugehörige Datenmodell in einer Implementierung sollte im Gegensatz zur aktuellen Implementierung der originale Eingabetext dennoch persistiert werden, und nicht nur der daraus abgeleitete reguläre Ausdruck.

\textbf{Begründung:}
Trotz der Reduzierung des Einsatzes von Wildcards und Mustern durch \hyperref[subsec:req-type-specific-matching]{A-ARTIFACT-TYPESIDENTIFICATION} bleibt die Notwendigkeit, komplexere Vergleiche durchzuführen als nur Gleichheitsprüfungen von Attributen.
Eine Erweiterung durch die Erfassung von Gruppen und Bereitstellung als Variablen kann optional erfolgen.

\textbf{Fit-Kriterium:}
Attributwerte können reguläre Ausdrücke enthalten.
Diese werden bei der Artefaktidentifikation korrekt interpretiert.

\subsection{A-MANUAL-MODIFICATION: Modifikation des Graphen}\label{subsec:req-manual-format-modification}

\textbf{Ziel:}
Das neue Korrelationssystem muss ein Modifikationsformat bereitstellen, das die vollständige Manipulation der Graphstruktur ermöglicht.
Dies umfasst die Erstellung, Änderung und Löschung aller Knotentypen sowie die Verwaltung ihrer Kantenbeziehungen.
Die Selektion von Knoten muss ausschließlich über ihre eindeutige Typ-ID-Kombination erfolgen, um Konsistent mit \hyperref[subsec:req-node-id-type]{A-NODE-IDENTIFICATION} zu sein.
Das Format muss so gestaltet sein, dass es sowohl für manuelle Bearbeitungen als auch für automatische Prozesse geeignet ist.

\textbf{Begründung:}
Dieses Modifikationsformat dient als einheitliche Schnittstelle für alle Graphmodifikationen und sorgt dafür, dass alle Änderungen am Graphen garantiert konsistente Ergebnisse unabhängig vom Änderungsursprung erzeugen.

\textbf{Fit-Kriterium:}
Das System stellt eine Menge an Operationen für die Modifikation von allen Graphkomponenten bereit.
Knoten werden ausschließlich über ihre Typ-ID-Kombination referenziert.
Das gleiche Format wird sowohl für manuelle Eingaben als auch für automatische Prozesse verwendet.

\subsection{A-MODIFICATION-YAML-FORMAT: Manuelles Modifikationsformat}\label{subsec:req-manual-format-modification-for-real}

\textbf{Ziel:}
Das manuelle Modifikationsformat muss als \acrshort{yaml}-basierte Schnittstelle zu dem in \hyperref[subsec:req-manual-format-modification]{A-MANUAL-MODIFICATION} geforderten Modifikationssystem dienen.
Das Einführen eines Graphens in das Korrelationssystems erhöt die Komplexität mit diesem umzugehen.
Die Syntax des Formats muss die komplexen Graphoperationen daher so abstrahieren, dass sie für Benutzer einfach handhabbar sind und dennoch die volle Funktionalität erhalten bleibt.

\textbf{Begründung:}
Diese Anforderung ermöglicht notwendige manuelle Korrekturen der automatischen \acrshort{cpe}-Erkennung, welche eine der Hauptaufgaben des aktuellen Korrelationssystems ist.
Da die Kombination aus Typ und Id jedes Knotens im Graphen eindeutig ist (s. \hyperref[subsec:req-node-id-type]{A-NODE-IDENTIFICATION}), wird das Auffinden von Einträgen in dem \acrshort{yaml}-Format deterministisch sein, was \hyperref[subsec:c-11-finding-yaml-entries]{C-11} löst.

\textbf{Fit-Kriterium:}
Jede Graphmodifikation kann über das \acrshort{yaml}-Format ausgedrückt werden und Benutzer können das Format ohne tiefe technische Kenntnisse des Graphmodells effektiv nutzen.
Es existiert eine bidirektionale, deterministische Mapping-Logik zwischen \acrshort{yaml}-Einträgen und Graph-Elementen.

\subsection{A-REASON-RATIONALE: Maschinenlesbare Entscheidungsdokumentation}\label{subsec:req-reason-format}

\textbf{Ziel:}
Das System muss in den strukturierten Metadaten der Knoten und Kanten Dokumentationseigenschaften als deren Bestandteil speichern.
Kanten müssen standardisierte Begründungsfelder für Relationen enthalten, Knoten müssen produktbezogene Informationen wie Beschreibungen und Referenzen in maschinenlesbarer Form ablegen können.
Diese Daten müssen über standardisierte Schnittstellen abfragbar sein.
Trotz der Maschinenlesbarkeit der Attribute dürfen diese nicht zu restriktiv in ihrem Format sein, denn eine Möglichkeit, Dokumentation anzulegen wird nur dann von einer Person effektiv regelmäßig durchgeführt, wenn diese einfach zugänglich ist.

\textbf{Begründung:}
Hiermit soll die mangelhafte Dokumentation im bestehenden System (\hyperref[subsec:c-05-reason-not-good-enough]{C-05}) erweitert werden und eine bessere Verarbeitung dieser Daten im System und Darstellung auf Benutzeroberflächen ermöglichen.

\textbf{Fit-Kriterium:}
Alle angesprochenen Metadaten sind in strukturierter Form gespeichert.

\subsection{A-GENERATION-FRAMEWORK: Generierungsframework für dynamische Daten}\label{subsec:req-generated-data}

\textbf{Ziel:}
Diese Anforderung verlangt vom neuen System, dass die Erzeugung eines effektiven Korrelationsdatensatzes in einen zweistufigen Prozess aufgeteilt wird.
Zunächst sollen mit den Datensätzen der \metaeffektsp die Knotenpunkte für alle bekannten \acrshortpl{cpe}, MS Product Ids und \acrshort{eol}-Ids automatisch erzeugt werden.
Wenn anwendbar, wie bei dem Datensatz \enquote{purl2cpe} von scanoss\footnote{\url{https://github.com/scanoss/purl2cpe}}, sollen die Datenquellen Kanten zwischen den Knoten anlegen.

Auf diesem Graphen soll optional in einem zweiten Schritt mit manuellen Modifikationen aufgebaut werden können.
Die generierten Daten müssen in dem sich ergebenden Graphen klar als solche mit ihrer Quelle gekennzeichnet und isoliert auslieferbar sein.

\textbf{Begründung:}
Im alten Korrelationssystem liegen die generierten \acrshort{yaml}-Dateien gleichwertig neben den manuell gepflegten Daten (\hyperref[subsec:c-08-generated-correlation-data]{C-08}).
Da ihre Existenz nicht von Beginn an geplant war, wurden das bisherige Selektor-System und die weiteren Matching-Regeln nicht für die besonderen Anforderungen die diese Quellen mit sich bringen ausgelegt.
Das neue System soll diese Datenquellen von Beginn berücksichtigen.

Zudem ermöglicht dieses System das Teilen des automatischen Anteils gemäß \hyperref[subsec:c-09-sharing-of-public-data]{C-09}.

\textbf{Fit-Kriterium:}
Im Erzeugungsprozess des Korrelationsdatensatzes ist die automatische Generierung von Daten berücksichtigt und Mechanismen werden dafür explizit bereitgestellt.
Generierte und manuelle Daten können klar voneinander getrennt angewandt werden.
Der generierte Anteil ist quellentransparent als solcher gekennzeichnet und kann einfach exportiert werden.

\subsection{A-YAML-FILES-ORGANIZATION: Organisation der \acrshort{yaml}-Dateien}\label{subsec:req-yaml-file-organization}

\textbf{Ziel:}
Manuelle Korrelationseinträge sollen noch immer in einer hierarchischen Verzeichnisstruktur organisiert werden.
Es soll für die neuen Dateien mindestens eine Auftrennung in Verzeichnissen nach Typ und/oder Ökosystem geben, und weiterführend können die einzelnen Hersteller- oder Produktgruppen jeweils in getrennten Dateien aufgeführt werden.

\textbf{Begründung:}
Damit sollen die Herausforderungen der tausenden Zeilen langen \acrshort{yaml}-Dateien wenigstens initial adressiert werden (\hyperref[subsec:c-04-groe-und-unubersichtliche-yaml-dateien]{C-04}).

\textbf{Fit-Kriterium:}
Logische Gruppierung nach den genannten Kriterien.

\subsection{A-OLD-FILES-TRANSFER: Überführung von Einträgen aus dem alten Korrelationsdatensatz}\label{subsec:req-current-dataset-conversion}

\textbf{Ziel:}
In der Transitionsphase vom alten Korrelationssystem zum neuen muss ein paralleler Betrieb beider Systeme ermöglicht werden.
Es soll zunächst explizit kein Konvertierungsprozess spezifiziert werden, mit dem alte Einträge semantisch äquivalent ins neue Format überführt werden können.
In dieser Phase werden alte Einträge in das neue Format überführt werden, um das alte langsam zu ersetzen.

\textbf{Begründung:}
Gewährleistet kontinuierlichen Betrieb während der Transition.
Konvertierung wird in den Ausblick der Arbeit mit aufgenommen.

\textbf{Fit-Kriterium:}
Beide System können unabhängig voneinander eingesetzt werden.
Um alle alten Einträge in das neue Format übersetzen zu können muss sichergestellt werden, dass der neue Graph mindestens die Fähigkeiten des alten Formats abdeckt.

\subsection{A-PERFORMANCE-OPTIMIZATION: Performance des neuen Korrelationsformats}\label{subsec:req-correlation-format-performance}

\textbf{Analyse:}
Das alte Korrelationssystem wurde bei unterschiedlichen Inventar-Größen mit dem aktuellen Datensatz mit 6400 Korrelationseinträgen getestet.
Die Zeiten werden in \autoref{tab:old-correlation-performance} angegeben.

Aus diesen Messwerten sind zwei Phasen erkennbar.
Zunächst läuft eine konstante Startzeit (bei dem getesteten Datensatz etwa 0,8-0,9 Sekunden), die unabhängig von der Anzahl der Artefakte im Inventar auftritt, in welcher die Korrelationseinträge aus dem \acrshort{yaml} ausgelesen werden.
Danach steigt die Laufzeit annähernd linear mit der Anzahl der Artefakte an, der Zeitbedarf pro zusätzlichem Artefakt ist also relativ konstant.

Da der Korrelationsschritt bisher bei weitem nicht der zeitintensivste Schritt in der Schwachstellenanalyse ist, ist die Optimierung nie eine kritische Anforderung gewesen.

\begin{table}[h!]
    \centering
    \begin{tabular}{l r r r r}
        \toprule
        \textbf{Artefakte} & \textbf{Ø [s]} & \textbf{Min. [s]} & \textbf{Max. [s]} \\
        \midrule
        0                  & 0,806          & 0,435             & 1,668             \\
        1                  & 0,897          & 0,453             & 1,919             \\
        500                & 2,033          & 1,504             & 3,277             \\
        1000               & 3,256          & 2,531             & 4,612             \\
        2000               & 5,920          & 4,543             & 6,979             \\
        3000               & 8,804          & 7,425             & 11,800            \\
        4000               & 9,433          & 8,780             & 10,338            \\
        5000               & 12,162         & 10,929            & 13,596            \\
        6000               & 13,609         & 12,461            & 15,381            \\
        \bottomrule
    \end{tabular}
    \caption{Gemessene Laufzeiten des alten Korrelationsformats bei unterschiedlichen Artefakt-Anzahlen.}
    \label{tab:old-correlation-performance}
\end{table}

\textbf{Ziel:}
Die Laufzeitperformance des neuen Systems muss linear mit der Artefaktanzahl skalieren und darf den doppelten Zeitbedarf des alten Systems (gemäß \autoref{tab:old-correlation-performance}) nicht überschreiten, mit einer maximalen gleichen Startzeit und einem linearen Anstieg mit Faktor <2.

\textbf{Begründung:}
Trotz erweiterter Funktionalität müssen für Produktivumgebungen geeignete Verarbeitungszeiten gewährleistet bleiben, um die Akzeptanz des neuen Systems auf Kundenseite sicherzustellen.

\textbf{Fit-Kriterium:}
Messbare Einhaltung der Performance-Grenzwerte bei Testinventaren, geringere Startzeit und linearer Anstieg mit Faktor <2 gegenüber Referenzwerten.

\subsection{A-DATA-DELIVERY: Auslieferung des Datensatzes}\label{subsec:req-correlation-data-delivery}

\textbf{Ziel:}
Der entstehende Korrelationsdatensatz muss als reiner Dateisystem-Export auslieferbar sein, ohne Abhängigkeiten von externen Infrastrukturkomponenten wie Datenbankkomponenten zu verlangen.

\textbf{Begründung:}
Es soll eine Kompatibilität mit den bestehenden Deployment-Prozessen und eine Minimierung der Betriebsanforderungen für eine praktische Nutzbarkeit gewähleistet sein.

\textbf{Fit-Kriterium:}
Vollständige Systemfunktionalität ohne zusätzliche installierte Komponenten neben der Software der \metaeffekt.
Auslieferung erfolgt ausschließlich durch Dateikopieroperationen.

\subsection{A-TECHNOLOGIES-USE: Zu verwendende Technologien}\label{subsec:req-lang-java}

Da die bestehende Software zum Schwachstellenmanagement der \metaeffektsp in Java geschrieben ist, muss auch die Implementierung des neuen Korrelationssystems in Java geschehen.
Zum Abspeichern des Graphen kommen die Bibliotheken infrage, die bereits auf dem Classpath des Projekts sind, um weitere Abhängigkeiten zu vermeiden.
Eine Datenbank würde sich zur performanten Abfrage und Transformation eignen.

\textbf{Ziel:}
Die Implementierung muss in Java 8 erfolgen und soll, wenn möglich, ausschließlich bereits im \metaeffekt-Classpath vorhandene Bibliotheken verwenden, ohne neue externe Abhängigkeiten einzuführen.
Die Einführung weiterer Open-Source Bibliotheken mit offenen Lizenzen ist theoretisch möglich, jedoch soll dies vermieden werden, falls möglich.

\textbf{Begründung:}
Berücksichtigung des Technologiestack und Vermeidung von Lizenz- und Wartungskomplexität.

\textbf{Fit-Kriterium:}
Keine zusätzlichen Projektabhängigkeiten, falls nicht anders vereinbart.
Lauffähigkeit in der vorhandenen JVM-Umgebung.

\subsection{A-QUALITY-METRICS: Qualitätsmetriken des Graphen}\label{subsec:req-graph-inner-consistency}

Um verifizieren zu können, dass der Graph sinnhaft strukturiert und in sich selbst konsistent ist, müssen einige Prüfungen darauf ablaufen können.
Diese prüfen mindestens die folgenden Eigenschaften:

\begin{enumerate}
    \item Zu jeder Repräsentation eines Produkts in dem Graphen darf maximal genau eine eindeutige Identifikation stattfinden.
    Dies bedeutet, dass jeder Knoten exakt eine Repräsentation vollständig modellieren sollte und diese nicht noch einmal von einem anderen Knoten partiell abgedeckt sein darf.
    Diese Metrik muss sowohl in dem Matching-Algorithmus, als auch als Modellierungsvorschrift umgesetzt werden.
    \item Eine Erweiterung zu dieser Metrik ist, dass pro Artefakt im Graphen nur ein einziger Produktknoten über eine \enquote{is}-Beziehung erreicht werden darf.
    Es muss also eine Prüfung auf die Erreichbarkeit und Überlappungen von Produktknoten von Artefaktknoten aus durchgeführt werden.
    % FIXME-KKL: das hier hat halt das Problem, dass wir den Graphen mit allen CPE, etc. füllen wollen und nicht immer ein zugehöriges Produkt erzeugt werden kann.
    \item Es darf keine losen Knoten geben, jede Repräsentation muss mindestens zu einem Produkt-Knotenpunkt verbunden sein um sicherzustellen, dass mit einer Identifikation auch ein Informationsgewinn stattfinden kann.
    \item Um einen sich selbst erklärenden Datensatz zu erhalten, müssen die Metadaten jedes Knotenpunktes aufgefüllt sein und die Kanten eine Begründung enthalten, warum sie so existieren.
    \item Der Graph soll sich selbst prüfen können: Nachdem ein Software-Inventar manuell geprüft wurde, sollen alle Quell-Artefakte mit ihren Identifikationen in einem separaten Datensatz abgelegt werden, um diese bei zukünftigen Änderungen am Graphen automatisch prüfen zu können.
    So kann die Identifikation eines bekannten Produktes sich nicht einfach so ändern, ohne, dass man es mitbekommen würde.
    \item Eine Prüfung auf zirkuläre Vererbungs-Referenzen soll sicherstellen, dass es immer einen spezifischeren Knoten gibt und keine Schleife entsteht.
\end{enumerate}
