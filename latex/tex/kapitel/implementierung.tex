\chapter{Konzeption und Implementierung des neuen Formats}\label{ch:modeling-implmentation}

Dieses Kapitel beschreibt die konzeptionelle Ausgestaltung und die initiale technische Umsetzung des neuen Korrelationssystems.
Nach der Modellierung des Korrelationssystems folgt die Implementierungsdarstellung und abschließend die Demonstration anhand konkreter Anwendungsfälle.


\section{Modell des neuen Korrelationssystems}\label{sec:model-modellierungsansatz}

Aufgrund der abgeleiteten Anforderungen und den Referenzfällen wird in den folgenden Unterkapiteln das neue Korrelationssystem mit der Struktur des Graphen, einem Modifikationsformat um den Graphen zu transformieren, dem \acrshort{yaml}-Format um die Modifikationen manuell durchführen zu können, und die Matching- und Durchquerungsregeln für die Auswertung von Repräsentationsauflösungen.

\subsection{Struktur des Graphen}\label{subsec:model-graph-struktur}
% Beschreibung der Knotentypen, Kantentypen und Relationen????

\subsection{Graphmodifikation}\label{subsec:model-graph-modification}

\subsection{\acrshort{yaml}-Format für Graphmodifikationen}\label{subsec:modell-graph-modification-yaml}

\subsection{Matching von Repräsentationen}\label{subsec:model-matching}
% Artefaktselektoren und Vererbungsregeln

\subsection{Durchquerungsregeln des Graphen}\label{subsec:model-traversal}
% Auswertungslogik für Relationen und Versionstransformation


\section{Implementierung und Integration}\label{sec:implementierung}

Das in \autoref{sec:model-modellierungsansatz} entworfene Modell des neuen Korrelationssystems sollte als Teil der Arbeit als Nachweis für die Realisierbarkeit als Java-Applikation in das bestehende Code-Repository implementiert werden.
Dazu werden zunächst die technischen und architekturellen Entscheidungen aufgeführt, dann die Implementierung vorgestellt.

\subsection{Technische Grundentscheidungen}\label{subsec:impl-tech-choices}
% Wahl von Java, SQLite, Bibliotheken

\subsection{Architekturübersicht}\label{subsec:impl-arch-overview}

\subsection{Implementierung}\label{subsec:impl-implementation}


\section{Beispielhafte Anwendung}\label{sec:beispiele-fertige-implementierung}

Zur Validierung des implementierten Systems werden die vier Referenzfälle aus \autoref{sec:reference-case-chapter} im neuen Format abgebildet.
Jedes Beispiel zeigt spezifische Aspekte des neuen Modells.

\subsection{Referenzfall 1: JavaScript-Paket}\label{subsec:example-js-package}
% \autoref{par:reference-case-walletconnect}

\subsection{Referenzfall 2: Java-Runtimes}\label{subsec:example-java-runtimes}
% \autoref{par:reference-case-java-runtimes}

\subsection{Referenzfall 3: Windows 10}\label{subsec:example-windows}
% \autoref{par:reference-case-windows}

\subsection{Referenzfall 4: Redis}\label{subsec:example-redis}
% \autoref{par:reference-case-redis}
