\chapter{Grundlagen}\label{ch:grundlagen}


\section{Schwächen und Schwachstellen}\label{sec:def-weakness-vulnerability}

% “vulnerability A weakness that can be exploited or triggered to produce an adverse effect. The inability to withstand adversity.” (Ross et al., 2022, p. 63)
% “weakness Defect or characteristic that may lead to undesirable behavior.” (Ross et al., 2022, p. 63)

\paragraph{Schwäche (Wakness)}
Der Begriff einer Schwäche in einer Software-Komponente bezeichnet einen intrinsischen Fehler, Defekt oder eine problematische Eigenschaft, die unter bestimmten Bedingungen in einer Software oder Hardware zu unerwünschtem Verhalten führen kann \autocite{Ross_Winstead_McEvilley_2022}.
Schwächen beschreibende grundlegende Fehlermuster, die unabhängig von spezifischen Implementierungen oder Angriffsszenarien existieren.

Das \acrfull{cwe}-Projekt\footnote{\url{https://cwe.mitre.org/about/index.html}}, unter der Leitung der MITRE Corporation, stellt einen großen Katalog solcher Software-Schwächen dar.
Als Community-getriebenes Projekt ordnet es Schwächen in einem hierarchischen System als \acrshort{cwe}-Einträge an, die verschiedene Abstraktionsebenen abdecken\ \autocite{wu2016cwe}.
Die Klassifikation reicht von abstrakteren Kategorien (\verb+CWE-20+: Improper Input Validation\footnote{\url{https://cwe.mitre.org/data/definitions/20.html}}) bis zu konkreteren Implementierungsfehlern (\verb+CWE-125+: Out-of-bounds Read\footnote{\url{https://cwe.mitre.org/data/definitions/125.html}}).

\paragraph{Schwachstelle (Vulnerability)}

Während Schwächen allgemeine Fehlermuster beschreiben, stellt eine Schwachstelle eine konkrete Ausprägung einer Schwäche in einem spezifischen System oder Produkt dar.
Sie beeinträchtigt die zentralen \acrshort{ciavuln}-Sicherheitsziele (Confidentiality (Vertraulichkeit), Integrity (Integrität) oder Availability (Verfügbarkeit)) und stellen eine Verletzung formeller oder impliziter Sicherheitsanforderungen dar.
Das \acrshort{cve}-Programm, betrieben durch die MITRE Corporation und unterstützt von US-Behörden wie der \acrfull{cisa}, bietet einen weltweit anerkannten Standard zur Identifikation und Beschreibung von öffentlich bekannten Schwachstellen.
Jede Schwachstelle erhält dabei eine eindeutige Kennung (z.B.\ \verb+CVE-2024-12345+) und wird in kooperation mit autorisierten \acrfull{cna} mit einer Beschreibung der Schwachstelle, Referenzen zu betroffenen Produktversionen, der zugrundeliegende \acrshort{cwe}-Klassifikation sowie einer Bewertung des Schweregrads klassifiziert\ \autocite{Ross_Winstead_McEvilley_2022, CveGlossaryCommonVulnerabilitiesAndExposures12mai2025}.
Die \acrshort{cve} Schwachstellen-Informationen werden über eine \acrshort{api} von der \acrshort{nvd} öffentlich zur Verfügung gestellt\footnote{\url{https://nvd.nist.gov/developers/vulnerabilities}}.

Im Rahmen dieser Arbeit werden ausschließlich \acrshort{cve} als Schwachstellen betrachtet, andere ähnliche Standards werden als \nameref{par:security-advisories} behandelt.

% noch erwähnen: in diesem kontext zählen nur CVEs als schwachstellen, nicht etwa osv und co.

\paragraph{Security Advisories}\label{par:security-advisories}

% OSV, CSAF, etc.
% ich mochte die tabelle aus “Table 2. Security vulnerability databases (VDBs).” (Bennouk et al., 2024, p. 865) A Comprehensive Review and Assessment of Cybersecurity Vulnerability Detection Methodologies


\section{Schachstelldatenbanken}\label{sec:security-vulnerability-databases}

% VDBs
% “3.3. Security Vulnerability Databases” (Bennouk et al., 2024, p. 864) A Comprehensive Review and Assessment of Cybersecurity Vulnerability Detection Methodologies
% auflisten von den unterschiedlichen lieferanten von schwachstellinfos und advisories und wie sie


\section{Software-Inventare}\label{sec:def-inventories}

% was ist ein software-inventar?
% was für ausprägungen gibt es? wie werden diese erzeugt?
% - cyclonedx
% - spdx
% - metaeffekt extractor

% “Specifying the precise inventory is so crucial for assessing vulnerabilities” (Bennouk et al., 2024, p. 857)


\section{Produktidentifikationsstandards und relevante Formate}

\subsection{Produkte und ihre Repräsentation}

% erst mal allgemein darüber reden worum es geht
% Name ≠ Identity
% Intrinsic vs extrinsic naming scheme
% warum gibt es so viele formate (jeder sein use-case)

\subsection{CPE}

\subsection{PURL}

\subsection{MSRC Product-Ids}

\subsection{EOL Product-Ids}


\section{Analyse bestehender Produkt-Mapping Algorithmen}

% “Table 3. Collected methods related to the similarity matching-based approach.” (Bennouk et al., 2024, p. 870) A Comprehensive Review and Assessment of Cybersecurity Vulnerability Detection Methodologies


\section{Interne Produktmodellierung der \metaeffektlg}


\section{Analyse des \metaeffektlg-Schwachstellenscanners}
