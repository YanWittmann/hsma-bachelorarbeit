\chapter{Einleitung}\label{ch:einleitung}


\section{Hintergrund und Motivation}\label{sec:hintergrund-motivation}

Moderne Software wird zunehmend aus einer Vielzahl kleinerer Bestandteile zusammengesetzt, was die Art und Weise, wie Software entwickelt, verteilt und betrieben wird, grundlegend verändert hat.
Ein großer Anteil dieser Komponenten basiert auf \acrfull{oss} und dient damit als Grundlage digitaler Infrastrukturen in nahezu allen Branchen:
Laut dem Open Source Monitor des Bitkom e.V.\ setzen 69\% der deutschen Unternehmen \acrshort{oss} in ihren Lösungen ein\ \autocite{OpenSourceMonitorWintergerst}.

Mit der wachsenden Abhängigkeit von \acrshort{oss} steigt jedoch auch die Verantwortung für deren sichere und nachvollziehbare Verwendung.
In dem bereits verabschiedeten \acrfull{cra} der Europäischen Union geht es um die Cybersicherheit von Produkten mit digitalen Elementen und berücksichtigt dabei auch \acrshort{oss}-Komponenten.
Ab dem 11.\ Juni 2026 sind Unternehmen verpflichtet, bekannte Schwachstellen in ihrer Software-Lösung und den verwendeten Komponenten zu identifizieren und Sicherheitsvorfälle zu dokumentieren.
Spätestens ab dem 11.\ Dezember 2027 sind alle Vorgaben des \acrshort{cra} verbindlich.
Eine aktiv ausgenutzte Schwachstelle ist nach dem \acrshort{cra} ein meldepflichtiger Sicherheitsvorfall und muss innerhalb von 72 Stunden an die zuständigen Behörden gemeldet werden\ \autocite{eu2024cra}.

Eine der am weitesten verbreiteten Quellen für Schwachstelleninformationen ist das \acrfullr{cve}-System, das als Bestandteil der \acrfull{nvd} des \acrfull{nist}\footnote{\url{https://nvd.nist.gov/vuln}} mit derzeit fast 300.000 Einträgen die umfangreichste öffentlich zugängliche Schwachstellendatenbank darstellt\ \autocite{nvd12mai2025dashboard}.
Diese große Anzahl an erfassten Schwachstellen und der kontinuierliche Zuwachs machen es in Kombination mit den kurzen Meldefristen des \acrshort{cra} unmöglich, den Prozess der Schwachstellensuche rein manuell abzubilden.
Für 65\% der Unternehmen ist eine manuelle Überwachung dieser Datenmengen durch den zeitlichen Aufwand und das Risiko eines menschlichen Fehlers unmöglich\ \autocite{OpenSourceMonitorWintergerst}, weshalb sie automatisierte \acrfull{vms} einsetzen.

Moderne und automatisierte \acrshort{vms} verarbeiten die erfassten Softwareinventare typischerweise in mehreren Schritten, um den firmeneigenen Softwarekatalog mit relevanten Schwachstelleninformationen aus verknüpften \acrfullpl{vdb} abzugleichen:
die Identifikation der eingesetzten Komponenten, die Erkennung potenzieller Schwachstellen, eine regelbasierte Bewertung der Ergebnisse und die Erzeugung von Ergebnisberichten\ \autocite{Idrissi_Sebai_Faroukhi_Mahouachi_2024}.

Die Zuordnung der eigenen Komponenten zu den Produktbezeichnern der jeweils abzugleichenden \acrshort{vdb} ist als erster Schritt bereits einer der herausforderndsten.
Die \acrshort{nvd} etwa stellt mit dem \acrfull{cpe}-System ein Schema zur Verfügung, um Softwareprodukte zu identifizieren und zu \acrshort{cve} zuzuordnen\ \autocite{Cheikes_Waltermire_Scarfone_2011}.
In der Praxis ist die korrekte Abbildung der eingesetzen Komponenten auf \acrshort{cpe}-Einträge jedoch mit Schwierigkeiten verbunden:
uneinheitliche Benennungen, Doppelungen, Rechtschreibfehler und mehr führen zu Fehlidentifikationen und unvollständigen Ergebnissen im darauf folgenden Schwachstellen-Scan\ \autocite{Sanguino_Uetz_2017}.
Dies macht es unmöglich, sich ausschließlich auf die Ergebnisse dieser automatischen Scans zu verlassen und erfordert eine manuelle Kontrolle der Ergebnisse, insbesondere der erschlossenen Komponenten-Abbildungen\ \autocite{Sanguino_Uetz_2017}.

Das Problem beschränkt sich jedoch nicht auf die \acrshort{nvd} mit ihren \acrshort{cpe}, auch Produktbezeichner aus anderen Quellen wie die \acrfull{osv}-Datenbank\footnote{\url{https://osv.dev}}, den \acrfull{msrc} Update Guides\footnote{\url{https://msrc.microsoft.com/update-guide}}, endoflife.date\footnote{\url{https://endoflife.date}} und weitere lassen sich nicht ohne weiteres gegenseitig zuordnen.

Die \metaeffekt\ entwickelt seit mehreren Jahren ein eigenes \acrshort{vms}, bei dem die Abbildung von Komponenten auf verschiedene Produktbezeichner ein zentrales Problem darstellt.
Dafür wurde bisher ein \acrshort{yaml}-basiertes \enquote{Korrelationsformat} verwendet, das manuelle Anpassungen an den automatisierte Zuordnungen ermöglicht.
Dieses Format stößt jedoch zunehmend an seine Grenzen, weshalb mit dieser Arbeit ein neues System entworfen und eingeführt werden soll, das das bestehende Verfahren ersetzt.


\section{Zielsetzung und Forschungsfrage}\label{sec:ziel-forschungsfrage}

\paragraph{Problemherleitung}

Der erste Schritt eines \acrshort{vms}, die Abbildung von Komponenten auf \acrshort{vdb}-spezifischen Repräsentationen\ \autocite{Idrissi_Sebai_Faroukhi_Mahouachi_2024}, lässt sich auf folgende Kernfrage reduzieren:

\begin{quote}
    Um Schwachstellenabfragen für eine Komponente durchführen zu können, muss ermittelt werden, unter welchen \acrshort{vdb}-spezifischen Repräsentationen diese Komponente noch geführt wird.
    Wie kann diese Zuordnung zuverlässig hergestellt werden?
\end{quote}

Diese Fragestellung präsentiert bei genauerer Untersuchung zwei Herausforderungen:

\begin{enumerate}
    \item Kein einzelnes Schema (z.\,B. \acrshort{cpe}) deckt alle Komponenten oder \acrshort{vdb} vollständig ab.
    Es ist also die Erfassung von Abbildungen der Komponenten in mehrere Schemata nötig um ein umfassendes Bild der Schwachstellen zu erhalten.
    \item Automatische Zuordnungen sind oft unvollständig oder inkonsistent, während manuelle Pflege einen großen Aufwand mit sich bringt.
\end{enumerate}

Basierend auf diesen und weiteren Erkenntnissen über die letzten Jahre hat diese Arbeit als Ziel, ein neues System für die \metaeffektsp zu entwickeln und zu implementieren, das das alte Korrelationsformat ablösen wird.
Das neue System soll eine Abbildung zwischen verschiedenen Produktrepräsentationen ermöglichen, sodass ausgehend von einer beliebigen Identifikation (z.\,B.\ einer internen Komponente oder einer \acrshort{cpe}) alle zugehörigen Bezeichner ermittelt werden können.

\paragraph{Forschungsfrage}

Dies führt zu der Forschungsfrage dieser Arbeit:

\begin{quote}
    Wie muss ein Produkt-zentrischer Graph strukturiert sein, der heterogene Repräsentationen von Software- und Hardwarekomponenten (wie \acrfullr{cpe} oder \acrfullr{purl}) verknüpft und
    dabei sowohl manuelle Korrekturen, als auch die automatische integration von importierten Daten aus Schwachstellendatenbanken ermöglicht?
\end{quote}


\section{Aufbau der Arbeit}\label{sec:arbeit-aufbau}

Diese Arbeit gliedert sich in sechs thematische Abschnitte, die von den theoretischen Grundlagen über die praktische Umsetzung zu der Auswertung der Ergebnisse führen:

\paragraph{\autoref{ch:grundlagen}: Grundlagen}
Zunächst werden die relevanten Produktidentifikationsstandards (\acrshort{cpe}, \acrshort{purl}, \acrshort{msrc}-Ids, \acrshort{eol}-Ids) analysiert und bestehende Mapping-Algorithmen evaluiert.
Darauf aufbauend wird die interne Produktmodellierung der \metaeffektsp sowie der aktuelle Schwachstellenscanner untersucht.

\paragraph{\autoref{ch:anforderungen}: Anforderungsanalyse}
Ausgehend vom bestehenden \acrshort{yaml}-basierten Korrelationsformat werden dessen Schwächen erfasst und daraus Anforderungen für das neue System abgeleitet.
Besonderes Augenmerk liegt dabei auf Datenkonsistenz, Automatisierbarkeit und Prüfbarkeit.

\paragraph{\autoref{sec:model-modellierungsansatz}: Konzeption}
Basierend auf den Anforderungen wird das neue Korrelationsformat entworfen.
Dies umfasst die Erstellung des Modells, die Spezifikation der Schnittstellen zu externen Datenquellen sowie die Integration in bestehende Prozesse der \metaeffekt.

\paragraph{\autoref{sec:implementierung}: Implementierung}
Auf Grundlage der Konzeption folgt die technische Umsetzung des Systems.
Anwendungsbeispiele veranschaulichen dabei die Funktionsweise davon.

\paragraph{\autoref{ch:evaluation}: Evaluation}
Eine methodische Überprüfung bewertet das System anhand zuvor definierter Kriterien.
Die Ergebnisse werden kritisch im Kontext der ursprünglichen Anforderungen diskutiert.

\paragraph{\autoref{ch:abschluss}: Abschluss}
Die Arbeit schließt mit einer Zusammenfassung der Erkenntnisse und einem Ausblick auf zukünftige Erweiterungsmöglichkeiten des Systems.

\bigskip

Um einen ersten Einblick in die Lösung der Forschungsfrage zu geben, veranschaulicht die Visualisierung in \autoref{fig:example-graph-title-page} beispielhaft einen Anwendungsfall im entwickelten Graphenmodell.
Die Konzeption und Implementierung dieses Modells sind Gegenstand der nachfolgenden Kapitel.

\clearpage
\thispagestyle{empty}
\vspace*{\fill}
\begin{figure}[h!]
    \centering
    \makebox[\textwidth]{\includesvg[width=1.3\linewidth, inkscapelatex=false]{bilder/example-graph-title-page}}
    \caption{Visualisierung eines Subgraphen des Korrelationssystems. Die Abbildung zeigt einen Subgraphen aus dem neuen Korrelationssystem, dessen Modell Gegenstand dieser Arbeit ist. Abgebildet ist die Beziehung zwischen zwei identifizierten Softwarekomponenten und ihre semantische Einbettung in das \acrshort{cpe}-System.}
    \label{fig:example-graph-title-page}
\end{figure}
\vspace*{\fill}
\clearpage
