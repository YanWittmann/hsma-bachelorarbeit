\chapter{Erhebung der Anforderungen an das neue Korrelationsformat}\label{ch:anforderungen}


\section{Methode zu Erhebung der Anforderungen}

Zur Herleitung der Anforderungen an das neue Korrelationssystem wird zunächst die proprietäre Produktmodellierung der \metaeffektsp beschrieben, die in den meisten Fällen als Ausgangspunkt der Produktidentifikation dienen wird.
Anschließend wird der intern entwickelten Schwachstellenscanners vorgestellt, und das darin bisher verwendete Korrelationsformat zur Zuordnung von Komponenten mit ihren anderen Repräsentationen.
Das bestehende Format wird anschließend auf seine Schwächen und Herausforderungen untersucht, woraus die Anforderungen an das neue Korrelationsformat abgeleitet werden.


\section{Interne Produktmodellierung der \metaeffektlg}\label{sec:metaeffekt-inventory-format}

Wie alle Datentypen im Inventar basieren auch Artefakte auf Key-Value-Pairs, einer frei definierbaren Zuordnung von Textschlüsseln zu Textwerten.
Dieses Format wurde gewählt, um möglichst einfach Änderungen und neue Felder einführen zu können, ohne Schemata definieren und ändern zu müssen.
Komplexere Werte als flache Texte werden z.B.\ automatisch als JSON-Objekte serialisiert und beim Einlesen wieder als Objekte deserialisiert.

Für Artefakte ist die zugrunde liegende Struktur mit einigen Basis-Feldern für alle Software-Ökosysteme die gleiche, jedoch unterscheidet sich je nach Ökosystem die tatsächliche Verwendung der Felder.
Jedes Ökosystem hat somit eine übliche Menge an Feldern, mit denen eine Komponente beschrieben wird, die typischerweise ausgefüllt werden.

Beispielhafte Artefakte aus dem Java/Maven-Ökosystem sind in \autoref{tab:inventory-artifact-entries} dargestellt.

\begin{table}[ht]
    \caption{Beispielhafte Artefakteinträge in einem Software-Inventar}
    \label{tab:inventory-artifact-entries}
    \centering
    \begin{tabular}{llll}
        \toprule
        \textbf{Id}                  & \textbf{Component}         & \textbf{Group Id}        & \textbf{Version} \\
        \midrule
        commons-codec-1.15.jar       & Apache Commons Codec       & commons-codec            & 1.15             \\
        commons-collections4-4.1.jar & Apache Commons Collections & org.apache.commons       & 4.1              \\
        slf4j-api-1.7.36.jar         & SLF4J                      & org.slf4j                & 1.7.36           \\
        log4j-api-2.14.0.jar         & Apache Log4j               & org.apache.logging.log4j & 2.14.0           \\
        \bottomrule
    \end{tabular}
\end{table}

Da das Datenmodell tabellarisch angeordnet ist und damit in Tabelleneditoren mit mehreren Tabellenblättern abbildbar ist, wird es häufig in menschenlesbarer Form als \texttt{.xlsx} oder ähnlichen Formaten exportiert.

\section{Analyse des Schwachstellenscanners der \metaeffektlg}

% “The results show that the method describe” (Sanguino and Uetz, 2017, p. 22) Sanguino_Uetz_2017
%   auf diese art und weise die shortcomings unseres scanners beschreiben
% vor allem problematisch wenn es keine CPE gibt weil dann oft doch eine gefunden wird



\section{Aktuelles Korrelationsformat}


\section{Schwächen und Herausforderungen des aktuellen Korrelationsformats}


\section{Auflistung der Anforderungen}

% “4.3. Data Types in Falcon” ([Cheramangalath et al., 2016, p. 6](zotero://select/library/items/4DG8G9J3)) ([pdf](zotero://open-pdf/library/items/QNYKYEH4?page=6&annotation=C7MKFAUG))
