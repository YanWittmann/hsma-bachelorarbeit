\section{Beispielhafte Anwendung}\label{sec:beispiele-fertige-implementierung}

Zur Validierung des implementierten Systems werden die vier Referenzfälle aus \autoref{sec:reference-case-chapter} im neuen Format abgebildet.
Jedes Beispiel zeigt spezifische Aspekte des neuen Modells.

\subsection{Referenzfall 1: JavaScript-Paket}\label{subsec:example-js-package}
\autoref{par:reference-case-walletconnect}

\begin{lstlisting}[style=yaml,caption={Produktmodellierung zu react-walletconnect},label={lst:new-correlation-walletconnect},basicstyle=\ttfamily\scriptsize]
- cpe: "cpe:/a:uniswap:web3-react_walletconnect"

- product: "NPM React Walletconnect"
  is:
    - cpe: "a:uniswap:web3-react_walletconnect"

- artifact: "npm react walletconnect"
  identification:
    - type: npm
      name: [ "web3-react_walletconnect", "walletconnect", "@web3-react/walletconnect" ]
  is:
    - product: "NPM React Walletconnect"
\end{lstlisting}

\begin{figure}[htbp]
    \centering
    \makebox[\textwidth]{\includesvg[width=1.3\textwidth, inkscapelatex=false]{bilder/example-graph-walletconnect}}
    \caption{Generierter Graph zu react-walletconnect}
    \label{fig:example-graph-walletconnect}
\end{figure}

\subsection{Referenzfall 2: Java-Runtimes}\label{subsec:example-java-runtimes}
\autoref{par:reference-case-java-runtimes}

\begin{lstlisting}[style=yaml,caption={Produktmodellierung zu Amazon Correto},label={lst:new-correlation-java-runtimes},basicstyle=\ttfamily\scriptsize]
- cpe: "cpe:/a:amazon:corretto"
  namespace: "corretto (artifact to cpe)"
- cpe: "cpe:/a:oracle:jdk"
  namespace: "corretto (artifact to cpe)"
- cpe: "cpe:/a:oracle:jre"
  namespace: "corretto (artifact to cpe)"

- artifact: "java runtime amazon-corretto base identification"
  namespace: "corretto (artifact to cpe)"
  identification:
    - type: java-runtime
      supplier: "AMAZON_CORRETTO"
  is:
    - product: "Amazon Corretto Java Runtime"

- product: "Amazon Corretto Java Runtime"
  namespaces:
    "corretto (artifact to cpe)":
      - source: [ version: "/1\\.1\\.6[^0-9](.*[^0-9])?0*9([^0-9].*)?/i" ]
        target: [ version: "1.1.6_009" ]
      - source: [ version: "/1\\.1\\.6[^0-9](.*[^0-9])?0*8([^0-9].*)?/i" ]
        target: [ version: "1.1.6_008" ]
  is:
    - cpe: "a:amazon:corretto"
    - cpe: "a:oracle:jdk"
    - cpe: "a:oracle:jre"
\end{lstlisting}

\begin{figure}[htbp]
    \centering
    \makebox[\textwidth]{\includesvg[width=1.3\textwidth, inkscapelatex=false]{bilder/example-graph-java-runtimes}}
    \caption{Generierter Graph zu Amazon Correto}
    \label{fig:example-graph-java-runtimes}
\end{figure}

\subsection{Referenzfall 3: Redis}\label{subsec:example-redis}
\autoref{par:reference-case-redis}

\begin{lstlisting}[style=yaml,caption={Produktmodellierung zu Redis},label={lst:new-correlation-redis},basicstyle=\ttfamily\scriptsize]
- cpe: "cpe:/a:redis:redis"
  metadata:
    description: "For developers, who are building real-time data-driven [...]"
    references:
      homepage: https://redis.io
      repository: https://github.com/redis/redis
- cpe: "cpe:/a:redislabs:redis"
- cpe: "cpe:/a:pivotal_software:redis"

- cpe: "cpe:/a:redis:redis-py"

- product: "Redis DB"
  is:
    - cpe: "a:redis:redis"
    - cpe: "a:redislabs:redis"
    - cpe: "a:pivotal_software:redis"

- product: "Redis Python Interface"
  is:
    - cpe: "a:redis:redis-py"

- artifact: "Redis DB"
  identification:
    - attributes:
        Id: [ "Redis", "redis-*/i" ]
        Component: "redis/i"
  is:
    - product: "Redis DB"

- artifact: "Redis Python Interface"
  identification:
    - type: python
      name: "redis"
  is:
    - product: "Redis Python Interface"
  is not:
    - product: "Redis DB"
\end{lstlisting}

\begin{figure}[htbp]
  \centering
  \makebox[\textwidth]{\includesvg[width=1\textwidth, inkscapelatex=false]{bilder/example-graph-redis}}
  \caption{Generierter Graph zu Redis}
  \label{fig:example-graph-redis}
\end{figure}

\subsection{Referenzfall 4: Windows 10}\label{subsec:example-windows}
\autoref{par:reference-case-windows}

\begin{lstlisting}[style=yaml,caption={Produktmodellierung zu Windows 10},label={lst:new-correlation-windows-10},basicstyle=\ttfamily\scriptsize]
- cpe: "cpe:/a:windows:media_player"
- cpe: "cpe:/o:microsoft:windows"
- cpe: "cpe:/o:microsoft:windows_10_21h2"
- cpe: "cpe:/o:microsoft:windows_10"

- ms: "11929"
- ms: "11931"

- eol: "windows"

- product: "Windows 10"
  is:
    - cpe: "o:microsoft:windows_10"
    - eol: "windows"
  is not:
    - cpe: "a:windows:media_player"
    - cpe: "o:microsoft:windows"

- product: "Windows 10 21H2"
  is:
    - cpe: "o:microsoft:windows_10_21h2"
    - cpe: "o:microsoft:windows_10"
      transform:
        - version: 21h2
    - eol: "windows"
    - ms: "11931"
  inherit:
    - product: "Windows 10"

- product: "Windows 10 21H2 32bit"
  is:
    - ms: "11929"
  inherit:
    - product: "Windows 10 21H2"

- product: "Windows 10 21H2 64bit"
  is:
    - ms: "11931"
  inherit:
    - product: "Windows 10 21H2"

- artifact: "Windows 10"
  identification:
    - type: operating system
      attributes:
        Id: [ "Microsoft Windows 10*", "Windows 10*" ]
  is:
    - product: "Windows 10"
  metadata:
    description: "The base microsoft Windows 10 identification"
    references:
      release information: https://learn.microsoft.com/de-de/windows/release-health/release-information

- artifact: "Windows 10 21H2"
  identification:
    - version: "10.0.19044*"
  is:
    - product: "Windows 10 21H2"
  inherit:
    artifact: "Windows 10"

- artifact: "Windows 10 21H2 32 bit"
  identification:
    - attributes:
        Architecture: "*32*"
  inherit:
    - artifact: "Windows 10 21H2"
  is:
    - product: "Windows 10 21H2 32bit"

- artifact: "Windows 10 21H2 64 bit"
  identification:
    - attributes:
        Architecture: "*64*"
  inherit:
    - artifact: "Windows 10 21H2"
  is:
    - product: "Windows 10 21H2 64bit"
\end{lstlisting}

\begin{figure}[htbp]
    \centering
    \makebox[\textwidth]{\includesvg[width=1.3\textwidth, inkscapelatex=false]{bilder/example-graph-windows-10}}
    \caption{Generierter Graph zu Windows 10}
    \label{fig:example-graph-windows-10}
\end{figure}
