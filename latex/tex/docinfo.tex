% -------------------------------------------------------
% Informationen und Einstellungen für Ihre Abschlussarbeit
%

% Sprache für das Dokument festlegen
\newcommand{\hsmasprache}{de} %de für Deutsch oder en für Englisch


% Abgabeform festlegen
% Bei einer digitalen Abgabe, wird das Dokument einseitig erzeugt und der Titel wird
% zentriert.
\newcommand{\hsmaabgabe}{digital} % Abgabe erfolgt für Fakultät I digital. Optionen hier sind für anderen Fakultäten: "papier" oder "digital".


% Flags für Veröffentlichung, Sperrvermerk
\newcommand{\hsmapublizieren}{opensource}   
% Wird einer Veröffentlichung zugestimmt?
% Optionen: 
% opensource = Druck der CC Lizenz mit By SA (Standard)
% hs = Veröffentlichung an der Technischen Hochschule und auf Hochschulservern
% stud = kein opensource und keine veröffentlichung auf den Hochschulservern
% vertraulich = Arbeit darf nicht veröffentlicht werden und erhält einen Sperrvermerk (Nur nach Absprache mit Betreuer setzen!)


\newcommand{\genderhinweis}{gender}     % Soll der Gender-Hinweis angezeigt werden? ja=gender, nein = nogender; Genderhinweis wird nur in deutscher Sprache angezeigt!


\newcommand{\hsmaquellcode}{sourcecode} % Verwenden Sie Quellcode in Ihrer Arbeit? ja=sourcecode, nein= nosourcecode

\newcommand{\hsmasymbole}{symbole} % Verwenden Sie viele Symbole in Ihrer Arbeit, welche in einem Symbolverzeichnis aufgeführt werden sollen? ja=symbole, nein= nosymbole


\newcommand{\hsmaglossar}{glossar} % Verwenden Sie Begriffserklärungen nicht Abkürzungen in Ihrer Arbeit? ja=glossar, nein= noglossar

\newcommand{\hsmatc}{tc} % Verwenden der Änderungsmarkierung. Änderungsmarkierung aktiv und eine Liste der Änderungen wird angezeigt = tc, Keine Änderungsmarkierung und keine Ausgabe der Änderungen = notc




% Titel der Arbeit auf Deutsch
\newcommand{\hsmatitelde}{Eindeutige Produktzuordnung zur Schwachstellenanalyse: Modellierung von Produkten zur Abbildung mit heterogenenen Schwachstellendatenbanken}

% Titel der Arbeit auf Englisch
\newcommand{\hsmatitelen}{AUTO-TRANSLATED: Unambiguous Product Mapping for Vulnerability Analysis: Modeling of Products for Integration with Heterogeneous Vulnerability Databases}

% Weitere Informationen zur Arbeit
\newcommand{\hsmaort}{Mannheim}        % Ort
\newcommand{\hsmaautorvname}{Yan}      % Vorname(n)
\newcommand{\hsmaautornname}{Wittmann} % Nachname(n)
\newcommand{\hsmaabgabedatum}{2025-07-21} % Datum der Abgabe in dem Format JJJJ-MM-TT

\newcommand{\hsmafirma}{\{metæffekt\} GmbH} % Firma bei der die Arbeit durchgeführt wurde
\newcommand{\hsmabetreuer}{Prof. Thomas Smits, Technische Hochschule Mannheim} % Betreuer an der Hochschule
\newcommand{\hsmazweitkorrektor}{Karsten Klein, \{metæffekt\} GmbH}   % Betreuer im Unternehmen oder Zweitkorrektor

\newcommand{\hsmafakultaet}{I}    % I für Informatik oder E, S, B, D, M, N, W, V
\newcommand{\hsmastudiengang}{IB} % IB IMB UIB CSB IM MTB (weitere siehe titleblatt.tex)
